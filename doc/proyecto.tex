\documentclass[12pt,spanish]{report}
\usepackage[spanish]{babel}
\usepackage[utf8x]{inputenc}
\usepackage{fancyhdr}
\usepackage{graphicx}
\usepackage{hyperref}
\usepackage{listings}
\usepackage[pdftex,usenames,dvipsnames]{color}
\parindent 2em
\parskip 2ex
\pagestyle{fancy}
\fancyhead[LO]{\leftmark}
\fancyhead[RE]{\rightmark}
\fancyhead[RO,LE]{\thepage}
\fancyfoot[RO]{Daniel García Moreno}
\fancyfoot[LO]{SSH sobre federación de identidad}
\fancyfoot[CO,CE]{}

\renewcommand{\headrulewidth}{0.6pt}
\renewcommand{\footrulewidth}{0.6pt}
\setlength{\headheight}{1.5\headheight}

\definecolor{GrisClarito}{gray}{0.9}

\lstset{% opciones
        basicstyle=\ttfamily\scriptsize,
        commentstyle=\color{blue},
        identifierstyle=\bfseries,
        backgroundcolor=\color{GrisClarito},rulecolor=\color{blue},
        numbers=left,
        numberstyle=\small\color{Gray},
        showstringspaces=false
}


\begin{document}



% mirar las comas
% mirar las segundas personas
\begin{abstract}

Cada vez está tomando más importancia la web, y actualmente están
apareciendo multitud de aplicaciones, ya sea por parte de empresas o
de instituciones como universidades, etc. Estas aplicaciones
normalmente requieren autenticación, y la mayoría de ellas basan dicha
autenticación en bases de datos locales.

Con el crecimiento de las aplicaciones, crece el número de usuarios de
estas, y por parte del usuario, crece el número de cuentas creadas
para diferentes aplicaciones. Para paliar este problema, nace el
Single Sing On(SSO), que proporciona una única cuenta, y un único
punto de autenticación, para diferentes aplicaciones web de una misma
entidad.

El siguiente paso natural, es el uso de aplicaciones de otras
entidades, y aquí es donde radica la importancia de la
\textbf{federación de identidad}.

La federación de identidad consiste en que una serie de entidades
\textbf{confían} en otras para la autenticación de los usuarios. Es
decir, que una aplicación de una entidad, acepta usuarios de otra
entidad. Además estos usuarios se autenticarán en su entidad, por lo
que la gestión de usuarios, contraseñas, y atributos, queda delegada a
cada entidad. Por tanto el usuario final tiene acceso a todas las
aplicaciones federadas, de todas las entidades que conforman la
federación de identidad.

Esto facilita enormemente la gestión de usuarios, por parte de las
entidades, puesto que tan solo tienen que gestionar sus propios
usuarios, y prestan servicio a usuarios de otras entidades, gracias a
que la autenticación es delegada.

La idea principal de este proyecto es llevar las facilidades que
proporciona la federación de identidad fuera del ámbito de la web,
más concretamente al ámbito del \textbf{SSH} (Secure SHell).

Para el caso del SSH, si un usuario tiene acceso a diferentes
máquinas, tendrá diferentes cuentas, y diferentes contraseñas que
recordar, almacenar y gestionar, con la problemática que eso conlleva.
Además sus contraseñas estarán en máquinas que no tiene por qué
controlar, por lo que si se compromete alguna de estas máquinas,
estará comprometida su contraseña.

Utilizando SSH sobre la federación de identidad, se pueden eliminar
estos problemas e incrementar la comodidad, tanto por parte de los
usuarios, como por parte de los administradores. Para poder acceder
por SSH, un usuario tendría que autenticarse en la federación, y una
vez autenticado, podrá entrar en todas las máquinas que ofrezcan el
servicio de SSH federado, sin necesidad de poner contraseña, basándose
en el mecanismo de clave pública y clave privada, y estando el
servidor en cualquier entidad de la federación.

Así pues, por parte del usuario, se tiene acceso a diferentes máquinas
necesitando recordar y gestionar una sola contraseña. Además esta
contraseña nunca se entrega a un servidor extraño, sólo a la entidad
de la cual procedes (a través de una páginas web segura), y en la cual
confías puesto que es la encargada de gestionar tu identidad.

Y por parte del administrador, se puede delegar la gestión de
usuarios, confiando en la federación. Automatizando la creación y
destrucción de cuentas, y sin necesidad de proporcionar ninguna
contraseña.

\end{abstract}


\tableofcontents

\chapter{Introducción}
\section{Objetivos}

Las facilidades que ofrece la federación de identidad son más que
evidentes, y podría ser interesante en muchos casos, tener estas
facilidades para otros servicios que requieran autenticación.

El principal objetivo de este proyecto es llevar las facilidades de la
gestión de identidad del ámbito de la web a otros servicios, como por
ejemplo el SSH. En este proyecto nos hemos centrado en integrar la
federación de identidad con el acceso por SSH, y puede servir como
prueba de concepto a la hora de llevar la autenticación por federación
de identidad a servicios diferentes de la web.

Las características más importantes de la federación de identidad, que
nos serán útiles en el SSH son:
\begin{enumerate}

    \item \textbf{Acceso a recursos de otras entidades}: La base de la
    federación de identidad es poder acceder a recursos de otra
    entidad con la misma cuenta con la que accedes a los recursos o
    servicios de tu propia entidad.

    \item \textbf{Gestión de identidad distribuida}: Al encargarse
    cada entidad de la federación de sus propios usuarios, y
    basándonos en las relaciones de confianza de la federación, se
    puede dar servicio a un mayor número de usuarios gestionando tan
    solo una pequeña cantidad de ellos. Esto puede crear algún tipo de
    duda, puesto que se pierde el control sobre los usuarios, pero no
    hay que olvidar que la federación es una red de confianza, donde
    cada entidad debe confiar en las demás, y para ello hay mecanismos
    seguros, como por ejemplo los certificados.

    \item \textbf{Unicidad de contraseña}: La federación de identidad
    nos brinda la posibilidad de acceder a diferentes servicios, que
    requieren autenticación, con la misma cuenta y la misma
    contraseña, y sin necesidad de replicar esta en los diferentes
    servicios, sino estando en tu propia entidad, incrementando así la
    seguridad de la misma, y la comodidad a la hora de cambiar de
    contraseña, o de nombre de usuario.

    \item \textbf{Login único}: También se busca implementar el Single
    Sing On(SSO) para el SSH sobre federación, de tal forma que un
    usuario sólo tenga que autenticarse una vez, y a partir de ahí,
    tener acceso, sin necesidad de introducir ningún tipo de
    contraseña, a todos los servidores SSH disponibles.

\end{enumerate}

Por otra parte, hemos elegido llevar la federación al servicio SSH
porque es ampliamente utilizado, además de que ofrece una gran
potencia y versatilidad, abriendo así la puerta a la utilización de
otros servicios de forma fácil.

Por ejemplo, en el ambiente académico, puede ser interesante dar
acceso a un servidor SSH a todos los alumnos de Informática, bien sea
para que utilicen un supercomputador, o para que tengan una cuenta
dónde hacer las practicas. Dentro del objetivo de este proyecto
entraría delegar la gestión de estos usuarios a la federación,
facilitar así el proceso, así como por parte del alumno, como por
parte del administrador de las máquinas.
\newpage
\section{Caso de uso}

    En el siguiente caso de uso se muestra el funcionamiento básico del
    sistema, así como una serie de detalles que serán explicados
    detalladamente en la sección ``Implementación y despliegue''
    [\ref{implementacion}].

    \textbf{Caso de uso del proyecto SSH sobre federación de identidad}:

    \begin{itemize}

    \item \textbf{Descripción}:
    
    La federación está pensada para aplicaciones webs, pero sería
    interesante poder utilizar estos mecanismos para aplicaciones que
    autentican de otra manera diferente.

    En el caso del ssh federado intentamos llevar el concepto de hacer
    login una sola vez, y en tu entidad, al acceso por ssh. Buscando poder
    acceder por ssh a diferentes máquinas sin tener que escribir usuario y
    contraseña, una vez nos hayamos autenticado.  A través de ssh se pueden
    hacer muchas más cosas, como por ejemplo túneles ssh, port-forwarding,
    etc.

    \item \textbf{Proceso de Autenticación}:
    \label{casouso}

    \begin{enumerate}

        \item Se accede a una página especifica, protegida tras un SP.

        \item El usuario se autentica en la federación, y puede ver la
        página.

        \item Esta aplicación web intentará conseguir la clave RSA publica
        del usuario a través de los datos que manda la federación.

        \item Una vez autenticado en esa aplicación web, el usuario puede
        acceder a las cuentas ssh federadas de las que disponga sin tener
        que introducir password.

    \end{enumerate}

    Se puede ver un esquema del funcionamiento de este proceso en la figura
    \ref{fig:casodeuso}

    \item \textbf{Proceso de Autenticación alternativo}:
    La clave publica que recibe la aplicación a través de la federación
    será la de la máquina habitual del usuario. En caso de estar utilizando
    otra máquina es posible utilizar otra clave temporalmente.

    \begin{enumerate}

        \item Se accede a una página especifica, protegida tras un SP.

        \item El usuario se autentica en la federación, y puede ver la página.

        \item En la aplicación web se introduce la clave publica RSA temporal,
        para esta sesión.

        \item Una vez autenticado en esa aplicación web, el usuario puede
        acceder a las cuentas ssh federadas de las que disponga sin tener que
        introducir password.

    \end{enumerate}

    \item \textbf{Implementación}:
    Para la implementación se ha optado por utilizar el mecanismo de acceso
    por clave publica-privada que nos ofrece el mismo protocolo ssh.
    Este mecanismo es el siguiente:
    El usuario crea un par de claves, para la máquina en la que se
    encuentra (ssh-keygen).

    Para dar acceso remoto sólo necesitamos conocer la clave publica
    (\$HOME/.ssh/id\_rsa.pub).
    
    Para poder acceder es necesario que el usuario disponga de su par
    privado.
    
    El servidor openssh, mira en el directorio personal del usuario, y
    busca en el archivo authorized\_keys (\$HOME/.ssh/authorized\_keys), antes
    de pedir password. Si encuentra alguna clave, intenta la autenticación
    por RSA, que es automática, sin petición de password. Por lo tanto
    nuestro objetivo es utilizar este servicio, pero en lugar de mirar en
    un archivo local, preguntaremos a un servidor remoto.
    
    \item \textbf{Requisitos}:

    Para poder acceder a cualquier máquina remota por ssh, en el servidor
    se debe poner el sshd parcheado. Además se debe crear una cuenta de
    usuario. Es recomendable el deshabilitar la posibilidad de cambiar el
    password, puesto que si se puede cambiar el password de la cuenta, es
    posible acceder a esta sin pasar por la federación. También es
    conveniente no permitir la creación, o borrar, los ficheros dentro de
    .ssh del home del usuario, por la misma razón que lo anterior.

    También será necesario definir un schema para la federación, que añada
    el campo ssh\_rsa\_public\_key, si queremos que el acceso sea lo más
    automatizado posible.

    \item \textbf{Temas a discutir}:

    \begin{enumerate}
        \item Tipo del servidor de claves, LDAP, base de datos, etc

        \item Posibilidad de cambiar password, y permisos en la cuenta previamente creada.
    \end{enumerate}

    \end{itemize}

    \begin{figure}[htp!]
        \centering
            \includegraphics[width=\textwidth]{img/casodeuso1.png}
            \caption{Caso de Uso}
        \label{fig:casodeuso}
    \end{figure}

    Para una mayor compresión de la figura \ref{fig:casodeuso} vamos a
    explicar cada paso de manera un poco más exaustiva:

    \begin{itemize}
        
        \item{Paso 1:} Antes de poder acceder a cualquier servicio de
        ssh federado, el usuario tendrá que autenticarse. Para ello se
        utilizan los mecanismos que ofrece la federación de identidad.
        Por tanto el usuario intentará acceder a una aplicación web,
        protegida tras un Service Provider (SP) de la federación de
        identidad.
        Este SP no tiene por qué pertenecer a ninguna de las entidades
        que forman la federación, sino que será un único servidor
        central.

        \item{Paso 2:} Según el funcionamiento de la federación, si el
        usuario no está aún autenticado en la federación de identidad,
        el SP, redireccionará al usuario hacia una aplicación WAYF
        (where are you from), dónde seleccionará su entidad de origen,
        y tras esto será redirigido al Proveedor de Identidad (IdP) de
        su entidad. En esta aplicación web será donde el usuario
        proporciona sus credenciales, siendo esta una operación
        segura, puesto que el IdP está gestionado por nuestra propia
        entidad, y por lo tanto, no estamos proporcionando nuestra
        contraseña a ninguna otra entidad.

        \item{Paso 3:} Una vez autenticado en la federación de
        identidad, el sistema redirigirá al usuario al SP al cuál
        quería entrar en un principio, pasándole a este los datos
        necesarios para asegurar la identidad del usuario.
        Aunque este proceso pueda parecer largo, para el usuario son
        un simple par de clicks, puesto que todo el trabajo se realiza
        automáticamente por la federación.

        \item{Paso 4:} La aplicación principal, tras el SP, recibirá
        entonces los datos necesarios del usuario, que le serán
        proporcionados por el IdP de la entidad de este. Y con estos
        datos creará una entrada en el servidor de claves, el cuál
        será consultado por los servidores ssh para verificar que el
        usuario esté autenticado en la federación.
        En este paso es donde se comienza a sacar la federación de
        identidad del ámbito web, teniendo un lugar dónde se
        almacenarían los usuario autenticados, con un sistema
        diferente al de cookies.

        \item{Paso 5:} El usuario ya está autenticado, y ahora puede
        entrar en uno o varios servidores ssh federados sin necesidad de
        escribir su contraseña. Por supuesto tendrá un tiempo límite,
        la sesión cumplirá dado un cierto tiempo.

        \item{Paso 6:} El servidor ssh tiene que saber si el usuario
        que intenta acceder está autenticado en la federación, por lo
        que consultará el servidor de claves, para confirmar que dicho
        usuario está autenticado, y además es quien dice ser.

    \end{itemize}

%TODO importancia del SSH, túneles, importación X

\newpage
\section{Antecedentes}

    La federación de identidad es un concepto relativamente nuevo, y
    por tanto hoy en día hay pocas federaciones funcionando.

    La federación de identidad de universidades andaluzas está
    naciendo ahora mismo, por lo tanto se está nutriendo de la
    experiencia de otras federaciones con algo más de tiempo.

    Una de las federaciones más activas en lo que se refiere a
    investigación y desarrollo de aplicaciones para la federación de
    identidad es la federación noruega, \textbf{feide} (figura
    \ref{fig:feide}).

    \begin{figure}[htp]
        \centering
            \includegraphics{img/feide.jpg}
            \caption{Caso de Uso}
        \label{fig:feide}
    \end{figure}

    El proyecto de SSH sobre federación de identidad nace a partir de
    un documento publicado el 20 de agosto de 2007 por la federación
    noruega.

    %\href{http://rnd.feide.no/content/feide-and-ssh-secure-shell}{http://rnd.feide.no/content/feide-and-ssh-secure-shell}
    \textbf{http://rnd.feide.no/content/feide-and-ssh-secure-shell}

    Este documento investiga como las credenciales de la federación
    de identidad pueden ser usadas para autenticar a diferentes
    servicios, como el SSH.

    A partir de esta idea se proponen diferentes formas de conseguir
    la autenticación de SSH sobre la federación de identidad.
    Este proyecto se basa en las ideas propuestas por la federación
    noruega, dando un paso más, e intentando automatizar al máximo el
    proceso, para dar una mayor comodidad al usuario, y también al
    administrador.

    Todos los métodos propuestos en este documento se basan en la
    utilización de un navegador web para obtener las credenciales
    de autenticación del servicio SSH. Primero el usuario entra en
    una página web del proveedor de servicio SSH (SP). Como no está
    autenticado aún, se le muestra una página por defecto de
    bienvenida. Luego el usuario se autentica a través del
    proveedor de identidad (IdP).

    Como se puede observar en el caso de uso \ref{casouso}, este es el
    método que hemos utilizado en el proyecto de SSH sobre federación
    de identidad. Hemos mantenido la misma idea de autenticación
    propuesta por la federación noruega.

    En este documento se propone colocar una página web, tras un SP,
    por cada servidor SSH federado que se ofrezca. En nuestro caso no
    hemos mantenido esta idea, puesto que complicaría un poco el
    despliegue del servicio, y además para nuestra implementación sólo
    es necesaria una página central para la autenticación.

    A continuación vamos a detallar las diferentes propuestas que se
    pueden encontrar en este documento, además comentaremos los pros y
    los contras de estas ideas.

    \begin{itemize}
        
        \item{One-time passwords. Contraseñas de un solo uso:}
        Este método consiste en generar una contraseña de un solo uso
        para el usuario. Este es un método sencillo en el cual cada
        vez que un usuario se autentica, se generaría una cuenta con
        una contraseña aleatoria, valida solo una vez.

        Para este caso se puede utilizar OPIE (one-time password in
        everything), que es una aplicación desarrollada por el
        laboratorio de investigación naval de los Estados Unidos de
        América. Se puede utilizar en el servidor web el ejecutable
        ``opie-client'' para generar las contraseñas para los
        usuarios.

        Con este sistema, la contraseña generada sólo será valida una
        vez. Después de esto no podrá ser usada otra vez para
        autenticarse en este servidor.

        Este método es de fácil implementación, pero tiene grandes
        inconvenientes. El principal inconveniente es que es muy
        incomodo para el usuario, ya que tiene que ir copiando y
        pegando la contraseña para poder acceder.

        Otro gran inconveniente es que se pierde la posibilidad de
        Single Sign On (SSO), puesto que por cada servidor SSH tendrás
        una contraseña diferente.

        En cambio solventa el problema de tener que introducir tu
        contraseña en un servidor extraño, puesto que la contraseña
        que se introduce es de un solo uso.

        \item{Credenciales de la federación:} 
        Otra posibilidad es proporcionar nuestras credenciales de la
        federación de identidad directamente sobre el servidor SSH,
        sin pasar a través del la autenticación web. Esta solución
        puede ser implementada usando un pequeño módulo PAM, que se
        conecta a la federación y autentica al usuario directamente
        con las credenciales que este ha proporcionado.

        Pero este método no es una solución viable, puesto que el
        nombre de usuario y la contraseña nunca debería pasar a través
        de un proveedor de servicio.

        \item{Clave pública:}
        Una mejor opción es autenticar al usuario usando criptografía
        de clave pública. Primero el SP puede intentar obtener la
        clave pública del usuario a través de los atributos recibidos
        del IdP.
        
        Si la clave pública no se proporciona por parte del IdP, se
        puede ofrecer al usuario la posibilidad de subir la suya a
        través de una interfaz web. La clave se almacenará en el
        fichero authorized\_keys, permitiendo al usuario autenticarse
        usando su clave privada.

        Hay que tener en cuenta que si no se borra la clave del
        fichero authorized\_keys, el usuario podrá entrar directamente
        sin tener que autenticarse en la federación de identidad. Esto
        se podría solventar con un proceso cron que limpiara los
        fichero cada cierto tiempo.

        Este método es en el cuál nos hemos basado de una manera más
        directa, puesto que ofrece al usuario una autenticación mucho
        más simple, y solventa la mayoría de los problemas.

        La aplicación web tras el SP que hemos desarrollado es igual a
        la descrita en el documento. En un principio intenta conseguir
        la clave pública a través de los atributos que proporciona el
        IdP, pero también ofrece la posibilidad de introducir una
        manualmente.

        Sin embargo, aunque solventa el problema de tener que recordar
        una clave, y otros problemas relacionados con el uso de
        contraseñas, no es una solución completa. Al estar ligada a un
        solo servidor SSH no es posible entrar en diferentes
        servidores con una sola autenticación, sino que habría que
        autenticarse en cada uno de ellos. Esa parte es la que aporta
        nuestro proyecto.

        \item{Java SSH aplet:}
        Otra opción propuesta en el documento de la federación noruega
        es el uso de un servicio web, tras un SP, que lanza un applet
        SSH en Java. Se puede ofrecer un login automático.

        En este caso se realiza la operación inversa a lo que se busca
        en el proyecto, pero con similar resultado. En un principio
        queremos llevar los beneficios de la federación de identidad
        fuera del ámbito web, pero buscando este objetivo hacemos lo
        opuesto, y es llevar otro ámbito, como es el del SSH a la web,
        mediante el uso de un applet de Java.

        Lo bueno de este método es que tiene acceso a cuentas SSH de
        manera más simple aún, puesto que funcionaría de igual manera
        que cualquier página web federada. Además sigue estando la
        posibilidad de entrar en diferentes servicios SSH, habiendo
        realizado la autenticación una sola vez.

        Lo malo es que no te da un acceso SSH completo, sino que te
        ofrece una shell, pero siempre a través del navegador, con las
        limitaciones que esto conlleva. Además de que no aporta nada a
        la idea de llevar la federación de identidad fuera de la web.

    \end{itemize}

    En conclusión podemos decir que hay algunas investigaciones
    previas sobre el concepto de SSH sobre federación de identidad, y
    que la idea de realizar este proyecto nace, en gran medida, a
    partir del documento publicado por la federación noruega.
    
    Por supuesto nuestro proyecto no se queda en una prueba de
    concepto, como es el caso del documento comentado, sino que se a
    partir de estas pruebas de concepto se implementa un sistema algo
    más completo y funcional, el cuál utiliza y mejora las ideas
    propuestas, dando así un nuevo punto de vista a la idea principal.

    
    \subsection{Otros antecedentes a destacar}
    
        Como se verá en el capitulo \ref{openssh}, para la
        implementación hemos optado por crear un parche para el
        servidor SSH openssh, para realizar la autenticación por clave
        pública a través de un servidor externo, además de por los
        ficheros authorized\_keys.

        Por ello es interesante comentar en esta sección un proyecto
        que realiza algo similar a nuestra implementación.

        %\href{http://dev.inversepath.com/trac/openssh-lpk}{http://dev.inversepath.com/trac/openssh-lpk}
        \textbf{http://dev.inversepath.com/trac/openssh-lpk}

        Este proyecto consiste en un parche para el servidor SSH
        openssh, que nos da la posibilidad de autenticar a los
        usuarios a través del sistema de clave pública, almacenando
        las claves en un servidor LDAP.

        ``The lpk patch allows you to lookup ssh public keys over LDAP
        helping central authentication of multiple servers. This patch
        is an alternative to other authentication system working in a
        similar way (Kerberos, SecurID, etc...), except the fact that
        it's based on OpenSSH and its public key code.''
        
        Para este proyecto buscábamos algo parecido, pero en un
        principio no nos queríamos restringir al uso de un servido
        LDAP. Aunque posteriormente nos hemos basado en este protocolo
        para proporcionar el servidor de claves.

        Sin embargo no nos hemos decantado por usar este parche, por
        su complejidad, y hemos decidido implementar uno nuevo, con
        las mínimas variaciones posibles, y que cubriera nuestro caso
        específico.

        El parche implementado, es fácilmente modificable para que
        permita diferentes servidores de claves, no solo servidores
        LDAP.


\chapter{Análisis del problema}
    \section{Necesidad de federación de identidad}
        \subsection{¿Qué es la federación de identidad?}

    La gestión de identidad centralizada fue creada para ayudar con el
    trato de usuarios y seguridad de datos, cuando el usuario accede
    desde dentro de la misma red, o al menos desde dentro del mismo
    dominio de control. Cada vez más, los usuarios están accediendo a
    sistemas externos, que están fuera de su dominio de control, y
    usuarios externos están accediendo a sistemas internos.

    La federación de identidad es un concepto nuevo, nacido a partir
    del uso masivo de la web, así como de la necesidad de
    interoperabilidad entre diferentes entidades que gestionan
    diferentes grupos de usuarios de maneras totalmente diferente.

    Mediante soluciones de Identidad Federada los individuos pueden
    emplear la misma identificación personal (típicamente usuario y
    contraseña) para identificarse en redes de diferentes
    departamentos, empresas o entidades. De este modo las empresas o
    entidades comparten información sin compartir tecnologías de
    directorio, seguridad y autenticación, como requieren otras
    soluciones (metadirectorio, Single Sign On, etc.). Para su
    funcionamiento es necesaria la utilización de estándares que
    definan mecanismos que permiten a las empresas compartir
    información entre dominios. El modelo es aplicable a un grupo de
    empresas o a una gran empresa con numerosas delegaciones y se basa
    en el "círculo de confianza" de estas, un concepto que identifica
    que un determinado usuario es conocido en una comunidad
    determinada y tiene acceso a servicios específicos.

    La federación de identidad, describe las tecnologías, estándares y
    casos de uso los cuales sirven para habilitar la portabilidad de
    información de identidad entre diferentes dominios de seguridad
    autónomos. El objetivo final de la federación de identidad es dar
    acceso seguro a datos, sistemas u otros dominios, a los usuarios
    de un dominio, y sin necesidad de redundancia de administración de
    usuarios.

    El uso de la federación de identidad puede reducir costes
    eliminando la necesidad de escalar el sistema de gestión de
    identidad. La federación de identidad puede incrementar la
    seguridad y reducir los riesgos habilitando a una organización a
    identificar y autenticar un usuario una vez, y luego usar esta
    información de identidad entre diferentes sistemas, incluso entre
    páginas de organizaciones externas. Puede mejorar la privacidad
    permitiendo al usuario gestionar qué información es compartida, o
    limitando la cantidad de información compartida. Además puede
    mejorar drásticamente la experiencia del usuario final eliminando
    la necesidad de registrar una cuenta nueva, o la necesidad de
    tener que hacer login de manera redundante.

    Empresas líderes en todo el mundo han desplegado la federación de
    identidad para acercarse a sus socios, mejorar el servicio de
    compra, acelerar la ejecución de negocios, recortar costes y
    complejidad de integración con otros servicios, y liberarse a sí
    mismos de la autenticación.

        \subsection{Partes en la federación de identidad}

    Un sistema de federación de identidad está formado por diferentes
    partes, o sistemas necesarios para dar el servicio. A continuación
    se detallan las partes más importantes de este tipo de sistema,
    junto con algunos ejemplos de posibles tecnologías que pueden
    desempeñar el papel.

            \begin{itemize}

            \item \textbf{Single Sing On (SSO)}

    Un sistema Single Sing On, es un sistema que permite simplificar
    el acceso por parte del usuario, y de las aplicaciones, a
    diferentes aplicaciones, que comparten los mismos usuarios, y por
    tanto el mismo sistema de autenticación.

    Este sistema nace de la necesidad de autenticar y autorizar al
    mismo grupo de usuarios, pero en diferentes aplicaciones. Por lo
    tanto se propone un sistema de autenticación único, de tal modo
    que todas las aplicaciones utilicen este mismo sistema, y
    mejorando así la experiencia del usuario final, ya que siempre
    introducirá sus credenciales en un sistema homogéneo,
    independientemente de la aplicación, y además la sesión permanece
    entre diferentes aplicaciones. Por parte de las aplicaciones que
    utilizan este sistema, evita la necesidad de tratar con la
    identidad de estos usuarios, ya que el sistema de Single Sing On
    es el que los autentica.

    SSO es un método de control de acceso que permite autenticar al
    usuario una vez, y conseguir acceso a diferentes recursos en
    diferentes sistemas software.

    En una infraestructura homogénea, o al menos en una en la que
    exista un único esquema de autenticación, o donde existe una base
    de datos centralizada de usuarios, el sistema SSO es una gran
    ventaja.

    En la federación de identidad, el Sigle Sing On no es algo
    necesario, pero sí recomendable. No es un requisito indispensable
    para montar un sistema de gestión de identidad federada, pero sí
    que facilita mucho el acceso por parte del usuario.

    Por lo tanto en la mayoría de los sistemas de federación de
    identidad se hace uso de SSO.

    Algunos ejemplos de software que puede servir como SSO:
    \begin{itemize}
        \item \textbf{OpenID} \href{http://openid.net/}http://openid.net/
        \item \textbf{PAPI} \href{http://papi.rediris.es}http://papi.rediris.es
        \item \textbf{Sun Access Manager}
        \href{http://www.sun.com/software/products/access_mgr/index.jsp}http://www.sun.com/software/products/access\_mgr/index.jsp
        \item \textbf{OpenSSO} \href{https://opensso.dev.java.net/}https://opensso.dev.java.net/
    \end{itemize}

            \item \textbf{Proveedor de identidad (IdP)}

    Un proveedor de identidad es un sistema que gestiona la identidad
    de los usuarios de una organización, y ofrece un servicio de
    autenticación a otras aplicaciones.

    En la práctica un Proveedor de Identidad es una aplicación web a
    la cual serán redirigidos todos los usuarios de una organización
    que quieran acceder a una aplicación federada, donde se
    autenticarán, y este IdP ofrecerá los datos de autenticación a las
    diferentes aplicaciones federadas.

    Es necesario que haya un IdP por cada organización participante en
    la federación de identidad, puesto que es la vía de autenticación
    de sus usuarios internos.

    Es aquí donde se hace importante el sistema SSO del que hemos
    hablado antes, ya que se puede enlazar el IdP con el sistema de
    SSO para ofrecer un sistema de autenticación único para todas las
    aplicaciones federadas, ya sean internas a la organización o
    externas, por lo que el usuario final tendrá un acceso homogéneo a
    todo el sistema federado.

    Algunos ejemplos de software que pueden dar el servicio de Proveedor
    de Identidad:
    \begin{itemize}
        \item \textbf{Shibboleth IdP} \href{http://shibboleth.internet2.edu/}http://shibboleth.internet2.edu/
        \item \textbf{SimpleSAMLPHP} \href{http://code.google.com/p/simplesamlphp/}http://code.google.com/p/simplesamlphp/
    \end{itemize}

            \item \textbf{Proveedor de servicio (SP)}

    Un proveedor de servicio es la otra parte de la federación. Todas
    las aplicaciones web federadas, deberán estar protegidas por un
    proveedor de servicio.

    El SP permite impedir el acceso a las aplicaciones, redirigiendo a
    los usuarios a sus correspondientes IdP, para que se autentiquen,
    y en ese caso permite el acceso, comunicándose con el IdP, que le
    proporcionará la información necesaria para la autorización, y el
    correcto funcionamiento de las aplicaciones.

    Un SP está estrechamente relacionado con un servidor web, y se
    comunica con los diferentes IdPs de la federación. Esta es la
    forma que tiene el sistema de pasar los datos de identidad de una
    organización a otra, de forma segura, a través del protocolo SAML.

    Cada aplicación web federada debería estar protegida por un SP,
    así pues puede haber más de un SP por cada organización.

    Dada la estrecha relación entre el IdP y el SP, el software
    necesario para dar este servicio se puede encontrar en las mismas
    páginas. Tanto Shibboleth, como simplesamlphp ofrecen tanto el
    servicio de IdP, como el de SP:

    \begin{itemize}
        \item \textbf{Shibboleth IdP} \href{http://shibboleth.internet2.edu/}http://shibboleth.internet2.edu/
        \item \textbf{SimpleSAMLPHP} \href{http://code.google.com/p/simplesamlphp/}http://code.google.com/p/simplesamlphp/
    \end{itemize}

    En teoría es independiente el software que se utilice como IdP, o
    como SP por las diferentes organizaciones, ya que todos deberían
    compartir el mismo protocolo, SAML.

            \item \textbf{Where are you from (WAYF)}

    El objetivo del servicio "Where are you from" (WAYF) es guiar al
    usuario a su propio Proveedor de Identidad (IdP). A veces es
    llamado "Identity Provider Discovery".

    Básicamente lo que hace es presentar al usuario una lista de
    Proveedores de Identidad y redirigir al navegador del usuario al
    IdP seleccionado.

    Algunos sistemas WAYF implementan otras funcionalidades
    adicionales, que mejoran la experiencia del usuario, y la
    facilidad de uso. Por ejemplo existen varios métodos para recordar
    o adivinar el IdP del usuario.

    Un ejemplo de software que implementa este servicio se puede
    encontrar:
    
    \begin{itemize}
        \item \textbf{Switch} \href{http://www.switch.ch/aai/support/tools/wayf.html}http://www.switch.ch/aai/support/tools/wayf.html
    \end{itemize}

    El WAYF es una parte importante en el sistema de la federación,
    puesto que es el elemento que conecta los Proveedores de
    Servicio (SP) con los Proveedores de Identidad (IdP). Cuando un
    usuario intenta acceder a una aplicación protegida tras un SP,
    este será redirigido al WAYF, donde el usuario seleccionará cuál
    es su organización, y en consecuencia el sistema WAYF redirigirá
    al usuario al Proveedor de Identidad pertinente.

            \item \textbf{Servicio de directorio (LDAP)}

    ``Un servicio de directorio (SD) es una aplicación o un conjunto
    de aplicaciones que almacena y organiza la información sobre los
    usuarios de una red de ordenadores, sobre recursos de red, y
    permite a los administradores gestionar el acceso de usuarios a
    los recursos sobre dicha red. Además, los servicios de directorio
    actúan como una capa de abstracción entre los usuarios y los
    recursos compartidos.''

    Realmente no es imprescindible el servicio de directorio para
    desplegar una arquitectura de identidad federada, pero normalmente
    siempre va ligada a este tipo de aplicaciones. Esta estrecha
    relación se debe a que últimamente se está imponiendo el uso de
    esta tecnología para la gestión de identidad, de manera local. Por
    lo tanto la gran mayoría de las organizaciones ya tendrán un
    servicio de directorio para gestionar sus usuarios.

    La mayoría de sistemas y aplicaciones para desplegar una
    federación de identidad, están desarrollados pensando en este
    hecho, por lo tanto facilita la tarea.

    Sin embargo, es posible utilizar bases de datos, y otros tipos de
    almacenamiento o gestión de usuarios, aunque la configuración será
    más compleja, y tal vez no sea tan óptima.


            \end{itemize}


    \section{Uso de la federación fuera de la web}

    \section{La importancia del ssh}
         (túneles, importación X ...)


\chapter{Solución propuesta}

    En este cápitulo, comentaremos la solución que hemos propuesto, y que
    hemos llevado a cabo para conseguir una autenticación a través de la
    federación de identidad para SSH.

    La solución propuesta, consta de varias partes, como se puede apreciar
    en la figura \ref{fig:casodeuso}

    \section{Parche para openssh (proceso de autenticación sshd)}
    \label{openssh}
        \subsection{SSO para SSH, posibilidad}
    \section{Servidor de claves, flexibilidad, LDAP, python...}
    \section{Aplicación de login.}
    \section{Ejemplo aplicación creación de cuentas.}
 %FIX por aquí


\chapter{Implementación y despliegue}
    \label{implementacion}
    \section{Cambios al openssh}

    El código del parche para el servidor SSH debe tocar lo mínimo
    posible, para tener seguridad a la hora de aplicar actualizaciones
    de openssh. Así pues, la idea principal y que ha guiado el
    desarrollo de esta modificación ha sido esta, y por lo tanto se ha
    creado un parche lo más simple posible, con las menores
    dependencias de librerías externas, y fácilmente ampliable,
    pensando en casos futuros.

    Lo primero que hay que hacer para hacer una mejora o una adaptación de
    un proyecto software libre, es descargarse el código, y empezar a
    estudiarlo, para ver en qué partes del mismo hay que encajar la nueva
    funcionalidad.

    Para descargar el código se ha usado el sistema de control de versiones
    que utilizan para este proyecto, CVS
    (\url{http://www.openssh.com/portable.html}), bajando la versión para
    linux, puesto que es el sistema sobre el cuál se ha desarrollado todo
    el proyecto.
    
    \begin{verbatim}

    export CVSROOT=anoncvs@anoncvs.mindrot.org:/cvs
    export CVS_RSH=/usr/bin/ssh
    cvs get openssh

    \end{verbatim}

    Una vez conseguido el código, hubo un proceso de estudio del mismo,
    para entender minimamente el funcionamiento de la aplicación, antes de
    empezar a tocar nada.

    Tras estudiar el código, y realizar varias pruebas, se localizó la zona
    de código dónde se realiza la autenticación del usuario, y se comienza
    a modificar este código para añadirle la funcionalidad que estamos
    buscando.

    El fichero fuente, dentro de openssh, que se encarga de la
    autenticación SSH es el \texttt{auth2-pubkey.c}. Concretamente la
    autenticación por clave pública se realiza en la función
    \texttt{int user\_key\_allowed(struct passwd *pw, Key *key)}, por lo
    tanto esta es la función que hay que modificar.

    Concretamente este trozo de código es el encargado de comprobar que
    en el servidor existe un fichero que contiene las claves públicas que
    tienen acceso.

    \begin{lstlisting}

    file = authorized_keys_file(pw);
    success = user_key_allowed2(pw, key, file);
    xfree(file);
    if (success)
        return success;

    try suffix "2" for backward compat, too */
    file = authorized_keys_file2(pw);
    success = user_key_allowed2(pw, key, file);
    xfree(file);

    return success;

    \end{lstlisting}

    La variable \textit{file} tiene la ruta hacia el fichero dónde están las claves
    públicas para acceder, así pues añadir a este código la idea de que si
    un usuario no puede acceder por el método estandar, se compruebe si
    está en la federación, es tan simple como:


    \begin{lstlisting}

    file = authorized_keys_file(pw);
    success = user_key_allowed2(pw, key, file);
    xfree(file);
    if (success)
        return success;

    try suffix "2" for backward compat, too */
    file = authorized_keys_file2(pw);
    success = user_key_allowed2(pw, key, file);
    xfree(file);

    if (success)
        return success;

    // try external file fed+ssh <danigm>
    if(options.usefed == 1){
        get_rsa_key_ldap(options.fedserver, options.fedport,
                pw->pw_name, rsa_key);
        debug("RSA_EXTERNAL_KEY: trying this -> %s\n",rsa_key);

        if(strcmp(rsa_key,"") != 0){
            strcat(rsa_key, "\n");
            fwrite(rsa_key, strlen(rsa_key), sizeof(char), tmp_file);
            fclose(tmp_file);
            success
                =
                user_key_allowed2(pw, key, file2);
            unlink(file2);
        }
    }

    return success;

    \end{lstlisting}

    De esta manera, si no se ha podido acceder de la manera convencional,
    si está puesta la opción de usar el acceso federado, se hace una
    llamada a la función \texttt{get\_rsa\_key\_ldap}, que devolverá una
    cadena con la clave pública de este usuario si está autenticado, y en
    tal caso se creará un fichero temporal que se usará para intentar
    autenticar.

    Este simple cambio, y alguno más para definir las posibles opciones
    configurables con el fichero de configuración, es lo que se cambiaría
    de verdad sobre el código del servidor ssh real, por lo que el riesgo
    de introducir fallos adicionales al código se reduce considerablemente.

    Por tanto se delega la responsabilidad de saber si el usuario está
    autenticado o no a la función \texttt{get\_rsa\_key\_ldap}. Esta
    función se puede modificar, y cambiar por cualquier otra que devuelva
    una cadena con la clave pública del usuario que se le está pasando como
    argumento. Por esto es fácil implementar la comunicación con otros
    sistemas que no sean un servicio de directorio, o por alguna función
    que haga alguna operación más.

    Ya que la complejidad de la autenticación se ha delegado en esta
    función, se ha implementado en un fichero aparte, que se introduce en
    las herramientas de compilación, \texttt{automake}, y se enlaza para
    que sea accesible desde el código anteriormente comentado.

    Lo que hace la función básicamente es realizar una consulta al servidor
    LDAP buscando al usuario en cuestión. Si existe este, se extraen los
    datos de clave pública y timeout. Inmediatamente después se comprueba
    que el timeout no esté cumplido, y en ese caso se devuelve la clave
    pública del usuario, y se termina. En caso de que el timeout esté
    cumplido, o que no se encuentre al usuario, se devuelve una cadena
    vacía.

    En realidad la función es bastante sencilla y simple, pues así lo hemos
    requerido, y este era nuestro objetivo. Donde radica la mayor
    complejidad es en el uso de las llamadas a LDAP en el lenguaje de
    programación C, puesto que el acceso no es trivial, y durante el
    desarrollo del proyecto, en la infraestructura de pruebas se ha
    modificado la librería de acceso openldap, cambiando la forma de
    acceso, y poniendo las funciones utilizadas en la primera versión como
    \texttt{deprecated}, por lo tanto se ha tenido que reescribir el código
    para utilizar las nuevas funciones.

    Aquí está el código realmente relevante de esta función:

    \begin{lstlisting}
    
    LDAPMessage *entry=ldap_first_entry(ld, msg);
    for( attr = ldap_first_attribute(ld, entry, &ber);
            attr != NULL; attr = ldap_next_attribute(ld, entry, ber))
    {
        vals = ldap_get_values_len(ld, entry, attr);
        if (vals != NULL)  {
            for(i = 0; vals[i] != NULL; i++) {
                val = vals[i];
                /* process the current value */
                if (strcmp(attr, timeattr) == 0){
                        strcpy(timeout, val->bv_val);
                    }
                if (strcmp(attr, attribute) == 0){
                        strcpy(rsa_key2, val->bv_val);
                        debug("1 %s:%s\n", %attr, %rsa_key2);
                    }
            }
            if (check_timeout(timeout)) {
                    strcpy(rsa_key, rsa_key2);
                    debug("2 %s:%s\n", %attr, %rsa_key);
                }else debug("\nTIMEOUT CUMPLIDO\n");
        }
        ldap_value_free_len(vals);
    }


    \end{lstlisting}

    \section{Aplicaciones federadas: ssh, useradd}

    Como se vió en \ref{login}, la aplicación de login tiene un cometido
    específico, y una funcionalidad claramente definida. Por lo tanto la
    implementación de la misma ha sido el objetivo más fácil dentro de este
    proyecto.

    Se ha elegido la tecnología PHP, por su rápidez para el desarrollo web,
    además de por su simple integración con todo tipo de servidores web, y
    en nuestro caso con apache2. Además, dado que se ha decidido utilizar
    un servidor de claves basado en un servicio de directorio, se ha
    buscado una tecnología que ofrezca simplicidad a la hora de realizar
    estos accesos, y PHP cumple todos los requisitos.

    Se ha descompuesto la aplicación en dos partes diferenciables, para
    seguir un modelo vista controlador. De tal forma que las operaciones
    con datos, y la lógica de la aplicación esté centralizada en
    \texttt{ssh\_backend.php}, y que la parte de muestra de información, y
    formateo de los datos con html esté en \texttt{ssh.php}.

    Explicaré paso a paso las funciones que se pueden encontrar en
    \texttt{ssh\_backend.php}.
    
    \begin{itemize}

    \item En primer lugar hay una serie de variables que definen las opciones del
    entorno dónde queramos desplegar la aplicación.

    \begin{lstlisting}

    $base_dn ='o=People,dc=us,dc=es';
    $servidor_ldap = "goonie.us.es";
    $puerto_ldap = 389;
    $bn = 'cn=admin,dc=us,dc=es';
    $pw = 'xxxxx';
    $minutes_timeout = 30;
    $shib_header = "HTTP_USERCERTIFICATE";
    $rsa_server_key_attr = 'sshpublickey';
    $rsa_server_timeout = 'schacuserstatus';

    \end{lstlisting}

    Están los datos para el acceso al servidor de claves, el tiempo valido
    de sesión, el atributo de shibboleth dónde vendrá el certificado desde
    el IdP, y los atributos en el servidor de claves, dónde se almacenarán
    los datos.


    \item función \texttt{modify}
    
    \begin{lstlisting}

    function modify($ds, $uid, $pubkey){
        global $base_dn;
        global $minutes_timeout;
        global $rsa_server_key_attr;
        global $rsa_server_timeout;
        // preparar los datos
        $timeout = $minutes_timeout * 60; //5 minutos
        $hoy = getdate();
        $timeout = $hoy[0]+$timeout;
        $dn = "uid=".  $uid .",".  $base_dn;
        $info[$rsa_server_key_attr][0] = $pubkey;
        $info[$rsa_server_timeout][0] = "schac:userStatus:us.es:timeout:" . $timeout;

        // anadir la informacion al directorio
        $r=ldap_modify($ds, $dn, $info);
        $h = getdate($timeout);
        
        echo '<p class="info">'._('Esta sesion de ssh es valida 
        hasta: ').$h["hours"].':'.$h["minutes"].':'.$h["seconds"].
        ' - '.$h["mday"].' '.$h["month"].' '.$h["year"].'</p>';

        return $r;
    }

    \end{lstlisting}

    Esta función modifica una entrada ya existente en el directorio,
    cambiando la clave pública y poniendo nuevamente el timeout.

    \item función \texttt{add}

    \begin{lstlisting}

    function add($ds, $uid, $sn, $cn, $pubkey){
            global $base_dn;
            global $minutes_timeout;
        global $rsa_server_key_attr;
        global $rsa_server_timeout;
            // preparar los datos
            $timeout = $minutes_timeout * 60; //5 minutos
            $hoy = getdate();
            $timeout = $hoy[0]+$timeout;
            $dn = "uid=". $uid .",". $base_dn;
            $info["objectClass"][0] = "person";
            $info["objectClass"][1] = "ldapPublicKey";
            $info["objectClass"][2] = "schacUserEntitlements";
            $info["uid"] = $uid;
        if($sn == '')
            $sn = $uid;
        if($cn == '')
            $cn = $uid;
            $info["sn"] = $sn;
            $info["cn"] = $cn;

            $info[$rsa_server_key_attr] = $pubkey;
            $info[$rsa_server_timeout] = "schac:userStatus:us.es:timeout:" . $timeout;

            // anadir la informacion al directorio
            $r=ldap_add($ds, $dn, $info);
        $h = getdate($timeout);
        echo '<p class="info">'._('Esta sesion de ssh es valida
        hasta: ').$h["hours"].':'.$h["minutes"].':'.$h["seconds"].
        ' - '.$h["mday"].' '.$h["month"].' '.$h["year"].'</p>';
            return $r;
    }

    \end{lstlisting}

    Añade una nueva entrada en el servidor de claves.

    \item función \texttt{get\_remote\_user}

    \begin{lstlisting}

    function get_remote_user(){
            $cad = $_SERVER["REMOTE_USER"];

            //$cad = $_SERVER["REMOTE_USER"];
            //partiendo el nombre, para crear danigm-us a partir de danigm@us.es
            $cad = $_SESSION['user'];
            $nado = explode('@', $cad);
            $name = $nado[0];
            $dominio = substr($nado[1], 0, stripos($nado[1], '.'));
            $name = $name.'-'.$dominio;
            return $name;
    }

    \end{lstlisting}

    Devuelve el uid del usuario que se utilizará como nombre de usuario
    para el acceso por SSH. Como cada uid no tiene por qué ser único, pues
    to que en diferentes organizaciones pueden existir usuarios con el
    mismo id, se concatena el dominio del mismo para poder diferenciar a
    usuarios con el mismo uid, pero pertenecientes a diferentes
    organizaciones. Se ha optado por utilizar el caracter '-' como
    separador, puesto que si se utiliza la '@', complicaría un poco el
    acceso por ssh.

    %TODO cuidado que se sale el texto
    \item función \texttt{get\_certificate}
    \begin{lstlisting}
    
    function get_certificate(){
        global $shib_header;
            $certificate = "";
        if (isset($_POST['key'])) {
            // Public key was not received from IdP. First check if user is posting its public key
            $certificate = $_POST['key'];
            }
            // Check if userCertificate attribute is set in SAML response
            else if (isset($_SERVER[$shib_header])) {
                    $certificate = $_SERVER[$shib_header];
                    $certificate = base64_decode($certificate);
            }
            // Trim certificate string
            $certificate = trim($certificate);
            $certificate = str_replace("\r","",$certificate);
            $certificate = str_replace("\n","",$certificate);

            return $certificate;
    }
    \end{lstlisting}

    Intenta conseguir la clave pública del usuario a través de las
    cabeceras que introduce el SP de shibboleth, a través de los datos que
    le manda el IdP.

    \item función \texttt{check\_certificate}
    \begin{lstlisting}

    function check_certificate($certificate){
        // Check if certificate syntax is correct
        if (substr($certificate, 0, 7) == "ssh-rsa" ||
           substr($certificate, 0, 7) == "ssh-dss")
                return true;
        else
                return false;
    }
    \end{lstlisting}

    Comprueba si un certificado es valido.

    \item función \texttt{get\_certificate\_used}
    \begin{lstlisting}

    function get_certificate_used($uid){
        global $rsa_server_key_attr;
        $timestamp = get_attr($uid, $rsa_server_timeout);
        $timestamp = split(":", $timestamp);
        $timestamp = $timestamp[count($timestamp)-1];
        $now = getdate();
        if ($now > $timestamp)
        return get_attr($uid, $rsa_server_key_attr);
        else
        return "";
    }

    \end{lstlisting}

    Mira en el servicio de directorio, para un usuario ya existente cuál es
    la última clave pública que se utilizó.

    \item función \texttt{get\_attr}
    \begin{lstlisting}

    function get_attr($uid, $attr){
            global $base_dn;
            global $servidor_ldap;
            global $puerto_ldap;
            global $bn, $pw;

            //Conectando con el ldap
            $ds=ldap_connect($servidor_ldap, $puerto_ldap)
                or die("No ha sido posible conectarse al servidor ".$servidor_ldap."");
            //Version del protocolo que vamos a usar
            ldap_set_option($ds, LDAP_OPT_PROTOCOL_VERSION, 3);
            //Bind como usuario, vamos a buscar, para ver si ya esta
            ldap_bind($ds, $bn, $pw) or
                die("No ha sido posible enlazar con el servidor".
                $servidor_ldap." con el usuario ".$bn."");

            //se guardan por uid, por lo que filtramos por este campo
            $filter = '(uid='.$uid.')';
            $resource = ldap_search($ds, $base_dn, $filter);
            $info = ldap_get_entries($ds, $resource);
        ldap_unbind($ds);
        if ($info["count"] == 0){
            return null;
        }
        else{
            return $info[0][$attr][0];
        }

    }
    \end{lstlisting}

    Busca en el servicio de directorio, para un usuario existente un
    atributo.

    \item función \texttt{doit}
    \begin{lstlisting}

    /**
     * Esta funcion mira en el servidor de claves, si este usuario esta ya
     * si esta modifica la clave, y el timeout
     * si no esta lo anade
    **/
    function doit($uid, $pubkey){
            global $base_dn;
            global $servidor_ldap;
            global $puerto_ldap;
            global $bn, $pw;

            //Conectando con el ldap
            $ds=ldap_connect($servidor_ldap, $puerto_ldap) or die("No ha sido posible conectarse al servidor ".$servidor_ldap."");
            //Version del protocolo que vamos a usar
            ldap_set_option($ds, LDAP_OPT_PROTOCOL_VERSION, 3);
            //Bind como usuario, vamos a buscar, para ver si ya esta
            ldap_bind($ds, $bn, $pw) or die("No ha sido posible enlazar con el servidor ".$servidor_ldap." con el usuario ".$bn."");

            //se guardan por uid, por lo que filtramos por este campo
            $filter = '(uid='.$uid.')';
            $resource = ldap_search($ds, $base_dn, $filter);
            $info = ldap_get_entries($ds, $resource);

            if ($info["count"] > 0){
                    //solo hay que anadir un pubkey, si es distinto
                    //y modificar el timeout.
                    $response = modify($ds, $uid, $pubkey);
            }else {
                    //nueva entrada
                    $cn = $_SERVER["HTTP_SHIB_PERSON_COMMONNAME"];
                    $sn = $_SERVER["HTTP_SHIB_PERSON_SURNAME"];
                    $response = add($ds, $uid, $sn, $cn, $pubkey);
            }

            ldap_unbind($ds);
            return $response;
    }
    \end{lstlisting}


     Mira en el servidor de claves, si este usuario está ya. Si está
     modifica la clave, y el timeout. Si no está lo añade.

    \item función \texttt{anadir\_usuario}
    \begin{lstlisting}

    /**
     * Codigos de error:
     * -1 certificado no valido
     * -2 No se ha podido completar la operacion
     * -3 El certificado esta en blanco, no lo ha establecido el idp
     *  1 Todo bien 
    **/
    function anadir_usuario(){
            // If $certificate is set, public key has been successfully received
            $certificate = get_certificate();
            if ($certificate != "") {
                    // Check if certificate syntax is correct
                    if(check_certificate($certificate)) {
                            $name = get_remote_user();
                            $response = doit($name, $certificate);
                            // Check if command executed successfully
                            if($response)
                                    return 1;
                            else 
                                    return -2;
                    } else {
                            return -1;
                    }
            }
            else return -3;
    
    \end{lstlisting}

    Esta función intenta conseguir el certificado pasado por el formulario
    de la página, si no lo consigue, intenta conseguirlo mirando los
    atributos pasados por el IdP.

    Una vez que tiene el certificado, comprueba su validez, y si es
    correcto, añade al usuario en el servidor de claves.

    \end{itemize}

    %TODO quitar estos cachos de código
    Por otro lado está la parte que muestra la información,
    \texttt{ssh.php}, cuyo funcionamiento es muy simple, y sólo hace
    llamadas al \texttt{ssh\_backend.php}.

    \begin{lstlisting}

<?php 

session_start();
//soporte para traducciones
$lang=$_GET['lang'];
if($lang=='en')
    $language="en_US.utf8";
else
    $language="es_ES.utf8";
putenv("LC_ALL=$language");
setlocale(LC_ALL, $language);
bindtextdomain("ssh", "./locale");
textdomain("ssh");

require("ssh_backend.php");

//Esta cosita sirve para poder cerrar una aplicacion, aunque
//las lazy sessions del shibboleth permitan el acceso
$GLOBALS['shib_Https'] = false;
$GLOBALS['shib_AssertionConsumerServiceURL'] =
"/federacion21.us.es/Shibboleth.sso";
$GLOBALS['shib_WAYF'] = "federacion21.us.es";

//si no esta autenticado, redirigimos al wayf
if( $_SERVER['HTTP_SHIB_IDENTITY_PROVIDER'] != "") {
        $_SESSION["user"] = $_SERVER['REMOTE_USER'];
}
else{
        $pageurl = "http://federacion21.us.es/protegido/ssh/ssh.php";
        $url = ($GLOBALS['shib_Https'] ? 'https' :  'http') .'://' .
                $GLOBALS['shib_AssertionConsumerServiceURL'] . "/WAYF/" .
                $GLOBALS['shib_WAYF'] .
                '?target=' . $pageurl;
        header("Location: ".$url);
}
?>

<html>
    <head>
        <title>SSH por federacion</title>
	<link rel="stylesheet" type="text/css" media="screen" href="style.css" />
    </head>
    <body>

	<div id="head"></div>
	<div id="main">
        <h2> <?php echo _('Bienvenido ') .
        htmlentities(get_remote_user()); ?></h2>
        <?php
            $var = anadir_usuario();
            $name = htmlentities(get_remote_user());

            if ($var == -1)
                echo '<p class="warning">'.
                _('No ha introducido un certificado valido.').'</p>';
            else if($var == -2)
                echo '<p class="warning">'.
                _('La operaci&oacute;n no se ha completado.').'</p>';

            else if($var == -3)
                echo '<p class="warning">'.
                _('No se ha facilitado un certificado, introduzcalo manualmente').'</p>';
            else
                echo '<p class="ok">'.
                _('Ahora puedes entrar por ssh en los servidores de la
                federaci&oacute;n, utilizando como nombre de usuario: 
                ').'<span class="user">'. $name .'</span></p>';
                
            $certificate = get_certificate_used($name);
            echo '<p>'._('Usando el certificado: ').'<br/>';
            $cert = str_split($certificate, 50);
	    echo '<div class="certificate">';
            foreach ($cert as $line){
                echo htmlentities($line)."<br/>";
            }
	    echo '</div>';
            echo '</p>';
        ?>

	<div class="info">
    <p>
<?php
            echo _('Si no est&aacute;s en tu puesto de trabajo,
            o no se encuentra tu clave, puedes proporcionar una manualmente,
            introducciendola en el siguiente campo. Mira en tu directorio
            $HOME/.ssh/id_rsa.pub');
?>
        </p>
        <p>
<?php
            echo _('Para utilizar el proceso autom&aacute;tico,
            ponte en contacto con tu proveedor de identidad.');
?>
        </p>
	</div>
	<div class="form">
            <?php display_form() ?>
	</div>
	</div>
    </body>
</html>

    \end{lstlisting}

    La aplicación de ejemplo de creación de cuentas, es una simple
    aplicación en php, que está basada en la anterior, por lo tanto
    comparte la mayor parte del estilo, y algo del código.

    Sin embargo, dado que es una aplicación de ejemplo, y su
    funcionamiento en teoría es muy simple, no se ha separado en
    varios ficheros, sino toda la aplicación está sobre un único
    fichero \texttt{useradd.php}.

    Esta aplicación hace uso de llamadas al sistema, para realizar
    comandos remotos en los servidores SSH registrados. Para que esto
    funcione, es necesario que el servidor donde se ejecute esta
    aplicación tenga las claves públicas de los demás servidores,
    puesto que se van a hacer llamadas a comandos remotos a través de
    SSH, y esto debería hacerse de manera automática.

    Veamos con más detalle cada una de las funciones que componen la
    pequeña aplicación de ejemplo.

    \begin{itemize}

    \item función \texttt{check\_user}

    \begin{lstlisting}

function check_user($uid, $servidor){
        if ($servidor == '')
                $servidor = "federacion21.us.es";

	exec("sudo ssh root@".$servidor." id ".$uid, $array, $retval);
	return $retval == 0;
}

    \end{lstlisting}

    Comprueba si un usuario tiene una cuenta ya creada en un servidor
    determinado. Para ello utiliza una llamada al sistema con el
    comando \texttt{ssh} hacia el servidor remoto, para ejecutar el comando
    \texttt{id}.

    \item función \texttt{useradd}

    \begin{lstlisting}

function useradd($uid, $servidor){
        if ($servidor == '')
                $servidor = "federacion21.us.es";

	$val = system("sudo ssh root@".$servidor." useradd -m -s /bin/bash -p xx -d /home/".$uid." ".$uid, $retval);
	return $retval;
}

    \end{lstlisting}

    Crea una cuenta para un usuario en un servidor remoto. Para ello,
    al igual que la función anterior utiliza el comando \texttt{ssh}
    para ejecutar el comando \texttt{useradd} en el servidor remoto.

    \end{itemize}
    
    El resto de la aplicación es trivial, y no tiene importancia
    comentarlo. Aquí se puede ver el código completo:

    \begin{lstlisting}

<?php 

$servers[0] = 'federacion21.us.es';
$desc[0] = "servidor1";
$servers[1] = 'federacion22.us.es';
$desc[1] = "servidor2";
$error = '';
$info = 'En esta p&aacute;gina podr&aacute;s solicitar la creaci&oacute;n
para el acceso ssh a los servidores';

session_start();
//soporte para traducciones
$lang=$_GET['lang'];
if($lang=='en')
$language="en_US.utf8";
else
$language="es_ES.utf8";
putenv("LC_ALL=$language");
setlocale(LC_ALL, $language);
bindtextdomain("ssh", "./locale");
textdomain("ssh");

//Esta cosita sirve para poder cerrar una aplicacion, aunque
//las lazy sessions del shibboleth permitan el acceso
$GLOBALS['shib_Https'] = false;
$GLOBALS['shib_AssertionConsumerServiceURL'] = "/federacion21.us.es/Shibboleth.sso";
$GLOBALS['shib_WAYF'] = "federacion21.us.es";

//si no esta autenticado, redirigimos al wayf
if( $_SERVER['HTTP_SHIB_IDENTITY_PROVIDER'] != "") {
        $_SESSION["user"] = $_SERVER['REMOTE_USER'];
}
else{
        $pageurl = "http://federacion21.us.es/protegido/ssh/useradd/";
        $url = ($GLOBALS['shib_Https'] ? 'https' :  'http') .'://' .
                $GLOBALS['shib_AssertionConsumerServiceURL'] . "/WAYF/" . $GLOBALS['shib_WAYF'] .
                '?target=' . $pageurl;
        header("Location: ".$url);
}

function useradd($uid, $servidor){
        if ($servidor == '')
                $servidor = "federacion21.us.es";

    $val = system("sudo ssh root@".$servidor." useradd -m -s /bin/bash -p xx -d /home/".$uid." ".$uid, $retval);
    return $retval;
}

function check_user($uid, $servidor){
        if ($servidor == '')
                $servidor = "federacion21.us.es";

    exec("sudo ssh root@".$servidor." id ".$uid, $array, $retval);
    return $retval == 0;
}

function get_remote_user(){
        $cad = $_SERVER["REMOTE_USER"];

        //$cad = $_SERVER["REMOTE_USER"];
        //partiendo el nombre, para crear danigm-us a partir de danigm@us.es
        $cad = $_SESSION['user'];
        $nado = explode('@', $cad);
        $name = $nado[0];
        $dominio = substr($nado[1], 0, stripos($nado[1], '.'));
        $name = $name.'-'.$dominio;
        return $name;
}
?>

<html>
    <head>
            <title>SSH por federacion</title>
            <link rel="stylesheet" type="text/css" media="screen" href="style.css" />
    <script type="text/javascript" src="js/jquery-1.2.1.min.js"></script>
    <script type="text/javascript" src="js/useradd.js"></script>

    </head>
    <body>

        <div id="head"></div>
        <div id="main">
            <h2> <?php echo _('Bienvenido ') . htmlentities(get_remote_user()); ?></h2>
            <?php
            if(isset($_GET['server'])){
                $s = $servers[$_GET['server']];
                $var = useradd(get_remote_user(), $s);
                if ($var != 0 && $var != 9)
                        $error = 'No ha sido posible crear la cuenta';
                else
                        $info = '<p>Se ha creado la cuenta correctamente en el servidor'.$s.'</p>';
            }
            //$var = useradd(get_remote_user(), 'federacion22.us.es');
            else{
                /*
                Mostrar lista de servidores
                disponibles, junto con enlace para crear cuenta.  Se crean por ssh, estando
                este servidor autorizado para entrar como root en los demas.  Quizas es
                mejor no hacerlo con usuario root, sino con un usuario que tenga permiso
                solo para el useradd.
                */
                echo "<h3>Servidores ssh ofrecidos por la Universidad de Sevilla</h3>";
                ?>
                <table>
                        <tr>
                                <th>Nombre</th>
                                <th>Descripci&oacute;n</th>
                                <th></th>
                        </tr>
                <?php
                for($i = 0; $i<count($servers); $i++){
                        echo '<tr>';
                        echo '<td>'.$servers[$i].'</td>';
                        echo '<td>'.$desc[$i].'</td>';
        if (check_user(get_remote_user(), $servers[$i]))
            echo '<td>Ya tienes cuenta en este servidor</td>';
        else
            echo '<td><a href="?server='.$uid.'" class="solicitar" id="'.$i.'" >solicitar cuenta</a></td>';
                        echo '</tr>';
                }
                echo "</table>";
        }
            ?>
            <div class="info">
                    <p class="warning">
                            <?php echo $error;?>
                    </p>
                    <?php echo $info ?>
            </div>
        </div>
    </body>
</html>

    \end{lstlisting}

    \subsection{Internacionalización}

    Puesto que estas aplicaciones son la parte que va a ver el
    usuario, se ha utilizado un sistema de internacionalización,
    \texttt{gettext}, para facilitar la traducción de las mismas. Así
    pues sería muy simple traducir estas aplicaciones y ofrecerlas en
    diferentes idiomas, según el idioma en el que esté configurado el
    navegador del usuario.

    De momento, y para este proyecto se ha realizado la versión en
    castellano, y una traducción al inglés.

    \section{Necesidades para montar la plataforma}

    Este proyecto se ha realizado con un software base determinado,
    puesto que depende de muchos componentes. En este apartado
    explicaremos cómo montar el proyecto desde cero, y tener una
    versión funcional.

    Como este proyecto depende de una federación de identidad,
    supondremos que pertenecemos a una organización de una federación,
    y por lo tanto tendremos diferentes SPs montados.

    Todo el código, el parche, y las dos aplicaciones en php, se
    pueden conseguir en la forja de rediris, en el proyecto AUPAAI 
    \url{https://forja.rediris.es/projects/aupaai/}.

    %TODO comentar cómo se consiguen los fuentes

    \begin{verbatim}

    svn checkout https://forja.rediris.es/svn/aupaai   
    cd aupaai/fed+ssh/

    \end{verbatim}

        \subsection{Cómo aplicar el parche}

        Para poder aplicar el parche, es necesario bajarse el código
        fuente de openssh, \url{http://www.openssh.org/portable.html}

        \begin{verbatim}

        export CVSROOT=anoncvs@anoncvs.mindrot.org:/cvs
        export CVS_RSH=/usr/bin/ssh
        cvs get openssh

        \end{verbatim}

        Una vez hecho esto, el siguiente paso es copiar el fichero
        .patch dentro del directio donde vayamos a compilar el
        servidor SSH, y aplicarlo.

        \begin{verbatim}
        
        cd openssh
        cp ../aupaai/fed+ssh/parche-openssh.patch .
        patch -p0 < parche-openssh.patch

        \end{verbatim}

        Por otra parte, para simplificar la instalación, y dado que la
        licencia BSD lo permite, está disponible el código parcheado,
        junto con las aplicaciones en la forja de rediris
        \url{http://forja.rediris.es/frs/download.php/775/openssh-federado.tar.gz}.

        Si todo ha ido bien, es hora de compilar, antes de nada, es
        necesario tener instalado el compilador, y todas las
        dependencias de openssh, además de las librerías openldap, ya
        que el parche hace uso de las mismas para comunicarse con el
        servidor de claves. Así pues se deben ejecutar los siguientes
        comandos.

        \begin{verbatim}

        ./configure --prefix=/ruta/de/instalacion
        make
        make install

        \end{verbatim}

        Una vez compilado, es hora de configurarlo para que funcione
        con nuestro sistema, cambiando el fichero \texttt{sshd\_config}

        \begin{verbatim}

        # External RSA key
        fedserver host.servidorclaves.com
        fedport 389
        usefed yes
        fedserver_root_dn "cn=admin,dc=us,dc=es"
        fedserver_root_pw password
        fedserver_base "o=People,dc=us,dc=es"
        fedserver_attr sshPublicKey
        fedserver_timeattr schacUserStatus

        \end{verbatim}

        Y con esto, ya estaría el servidor SSH parcheado, listo para
        funcionar de manera federada. Ahora sólo sería necesario crear
        las cuentas de los usuarios, que se puede hacer de forma
        dinámica, o estática, y sólo tendrán acceso cuando se
        autentiquen frente a la federación.

        \subsection{Cómo montar el SP y la aplicación web}

        La aplicación web de autenticación está hecha pensando en el
        SP de shibboleth 1.3. Por lo tanto será necesario tener un SP
        de shibboleth sobre un servidor web apache.

        Para funcionar, la aplicación requiere que estén instaladas
        las librerias de acceso a LDAP, para php, puesto que se va a
        comunicar con el servidor de claves, que en este caso será un
        servicio de directorio.
        
        Sobre un directorio protegido se despliega el directorio
        \texttt{ssh}, y ya estaría la aplicación de autenticación
        instalada en la federación.

        Cualquier usuario que quisiera acceder a algún servidor SSH
        federado, tendrá que entrar en esta página, de tal forma que
        escriba el registro correspondiente en el servidor de claves,
        y así el usuario tenga acceso a todos los servidores SSH
        federados.

        La aplicación es configurable, así que habrá que cambiar unas
        variables en el fichero \texttt{ssh\_backend.php}

        \begin{verbatim}

        $base_dn ='o=People,dc=us,dc=es';
        $servidor_ldap = "goonie.us.es";
        $puerto_ldap = 389;
        $bn = 'cn=admin,dc=us,dc=es';
        $pw = 'xxxx';
        $minutes_timeout = 30;
        $shib_header = "HTTP_USERCERTIFICATE";
        $rsa_server_key_attr = 'sshpublickey';
        $rsa_server_timeout = 'schacuserstatus';

        \end{verbatim}

        Con estas opciones se pueden configurar, el servidor de
        claves, el tiempo en minutos que va a durar una sesión de un
        usuario autenticado, la cabecera de shibboleth que mirará la
        aplicación para intentar conseguir la clave pública y los
        campos en el servidor de claves para almacenar la clave
        pública y el tiempo de sesión.

        \subsection{Cómo instalar el servidor de claves (openldap)}

        La última parte que hace falta para tener montado todo el
        proyecto, es el servidor de claves.

        Puesto que nos hemos decidido por un servicio de directorio,
        aquí es muy fácil la instalación, es más, hay diferentes
        opciones, comerciales, y no comerciales. Y dado que los
        campos a utilizar para almacenar los datos son variables,
        tanto en el servidor SSH parcheado, como en la aplicación de
        autenticación, no es obligatorio instalar ningún esquema,
        aunque sí recomendable.

        Explicaremos aquí cómo instalar un servidor de claves, en una
        distribución tipo Debian, y aplicarle los esquemas usados.

        Lo primero es instalar el servicio de directorio.

        \begin{verbatim}

        apt-get install slapd

        \end{verbatim}

        La configuración está en el directorio /etc/ldap/, y los
        esquemas para openldap hay que meterlos en /etc/ldap/schemas/.

        Descarga de los esquemas necesarios:

        Schema schac: 
        \url{http://www.rediris.es/ldap/esquemas/schac/schac-20061017-1.3.0b2.schema.txt}

        Schema openssh-lpk:
        \url{http://dev.inversepath.com/openssh-lpk/openssh-lpk\_openldap.schema}


        Se ha de mover los esquemas al directorio /etc/ldap/schemas/ y
        posteriormente modificar el fichero
        \texttt{/etc/ldap/slapd.conf} añadiendole las lineas

        \begin{verbatim}

        include         /etc/ldap/schema/openssh-lpk_openldap.schema
        include         /etc/ldap/schema/schac-20061017-1.3.0b2.schema

        \end{verbatim}

        
        \subsection{Todo en conjunto}

        Una vez que ya están todas las partes instaladas, hay que
        configurar tanto los servidores ssh parcheados, como la
        aplicación de autenticación para que se comuniquen con el
        servidor de claves públicas.

        %TODO poner algo más


\chapter{Despliegue de una maqueta en CONFIA}

    Para la prueba del proyecto, se ha utilizado la federación CONFIA
    (\url{http://confia.aupa.info/}),
    montada en fase de pruebas. CONFIA es la federación de identidad de las
    universidades andaluzas, es un proyecto pionero en España, que está
    comenzando ahora. Su objetivo principal es montar una infraestructura
    de identidad federada a nivel Andaluz, con todas las universidades
    andaluzas.

    Esta federación no está aún en producción, y utiliza usuarios de
    prueba, pero es prácticamente funcional. Actualmente utiliza el
    protocolo Shibboleth 1.3, y hay varios IdPs (Universidad de Sevilla,
    Universidad de Málaga, Universidad de Córdoba, etc), y varios SPs
    protegiendo diferentes aplicaciones en diferentes universidades.

    Además existe una infraestructura de metadatos y un WAYF global,
    que están en el CICA.

    Así pues, el proyecto se puede desplegar en este entorno, y en este
    entorno es en el que se ha desarrollado, sobretodo en la parte de la
    Universidad de Sevilla.

    Por parte de la Universidad de Sevilla, hay un IdP de Shibboleth, sobre
    un servidor Tomcat, que utiliza como sistema de SSO el SUN Access
    Manager.


    En la máquina \texttt{federacion21.us.es} hay un servidor Apache2
    protegido tras un SP de shibboleth.

\section{Máquinas involucradas}

    Hay varios servidores SSH parcheados funcionando dentro de la
    federación, a modo de ejemplo. Los servidores SSH parcheados se han
    colocado en el puerto 2222, para no interferir con los servidores SSH
    por defecto de cada máquina.

    \begin{itemize}

    \item \texttt{federacion21.us.es}, sistema operativo Ubuntu.
    \item \texttt{federacion22.us.es}, sistema operativo Ubuntu.
    \item \texttt{goonie.us.es}, sistema operativo debian. Máquina en la
    cuál se ha desarrollado todo el proyecto.

    \end{itemize}

    Tanto federacion21 como federacion22, son máquinas virtuales que corren
    sobre un entorno de virtualización XEN. Así pues el despliegue de
    pruebas no pone en peligro ninguna infraestructura delicada, puesto que
    estos servidores no contienen información sensible, ni prestan ningún
    servicio.

    Las aplicaciones web necesarias para la autenticación, y para la
    creación de cuentas están desplegadas en el servidor
    \texttt{federacion21.us.es}, tras el servidor web apache2, y protegidas
    con un SP.

    El servidor de claves es un servidor openldap, instalado en
    \texttt{goonie.us.es}, que almacena las claves públicas, los nombres de
    usuarios, y el tiempo de sesión, que luego consultan los servidores SSH
    parcheados.

    Dada esta infraestructura de pruebas implementada, hoy en día es posible
    acceder a estos servidores SSH parcheados con cualquier usuario de la
    Universidad de Málaga, que se autentique en la federación, o de la
    Universidad de Córdoba, etc.

\section{Servicio SSH federado en la Universidad de Sevilla}

    %TODO llevar esto a los agradecimientos
    La Universidad de Sevilla es una de las impulsoras del proyecto, siendo
    parte importante en él, y cobrando relevancia en las pruebas
    realizadas.

    A este proyecto se le ha facilitado acceso a todas las máquinas
    necesarias, y se le ha dado permiso para implantar un entorno de
    prueba, permitiendo así probar la viabilidad del sistema en un entorno
    más o menos real, y con diferentes organizaciones implicadas.

    De no ser así, habría sido imposible probar este proyecto, puesto que
    la infraestructura que hay detrás del proyecto implica tener muchas
    máquinas, en localizaciones diferentes, y con una base de usuarios de
    prueba.

    Además de los medios materiales, me han permitido el asistir a
    diferentes reuniones de trabajo del proyecto CONFIA, lo que me ha
    permitido compartir opiniones y difundir este proyecto, más allá del
    ámbito puramente académico.

\section{Pruebas}

    Se han realizado pruebas de acceso a diferentes servidores SSH, (tanto a
    federacion21, como a federacion22), con usuarios de la federación.

    En principio las pruebas se han realizado con usuarios de la misma
    Universidad de Sevilla. Puesto que el proveedor de identidad de la
    Universidad trabaja con una maqueta que aprovisiona los usuarios, no se
    trata de usuarios reales, en producción, sino que son ficticios, aunque
    basados en datos reales.

    Por otra parte, una vez el proyecto se consideró lo suficientemente
    maduro, se publicó el sistema a toda la federación, y diferentes
    universidades, desde sus propias instalaciones han probado y verificado
    el correcto funcionamiento del proyecto.

    Así pues se puede decir que se han realizado varias pruebas, con
    usuarios ficticios, y reales, aunque aún querían una serie de pruebas
    más estrictas, con mayor número de usuarios, y un nivel de concurrencia
    alto, para comprobar qué carga soporta, y comprobar que no ocurre
    ningún fallo no previsto.

\section{Futuros despliegues}

    \begin{itemize}

    \item El proyecto se ha desarrollado en paralelo con la federación de
    identidad de las universidades andaluzas, y en colaboración con CONFIA,
    por lo que es posible que cuando la federación se implante, alguna
    Universidad quiera dar este servicio. Por esa parte el proyecto tiene
    futuro dentro de las universidades andaluzas.

    %TODO explicar que el proyecto es interesante, que se ha
    %presentado en foros europeos, y es posible que se despliegue en
    %otras federaciones
    \item Al partir de un articulo publicado por la federación noruega, feide, es
    posible que esta esté interesada en probar el proyecto en su propia
    infraestructura, o quizás coger el código y adaptarlo a sus
    necesidades, mejorarlo, o realizar otro proyecto a partir de este. En
    todo caso espero que este proyecto sirva para otras federaciones.

    \item El supercomputador más potente de España, llamado "MareNostrum",
    está situado en Barcelona. Actualmente se está dando acceso a
    diferentes universidades a partes de este supercomputador, para
    aprovechar su capacidad de cálculo en la investigación universitaria.
    Este caso es para el que se ha desarrollado este proyecto.

    \item Además de para federaciones, este proyecto se puede utilizar en
    organizaciones o empresas que no pertenezcan a ninguna federación.
    Sería tan fácil como modificar el código de la aplicación php, y hacer
    que consiguiera los datos de otro lugar, ya sea una base de datos, un
    servicio de directorio, etc. Por esto podría ser interesante para
    cualquier empresa que tenga que lidiar con una gran cantidad de
    servidores, el no tener que cambiar contraseñas cada poco tiempo, ni
    tener que crear usuarios nuevos en cada máquina.

    \end{itemize}


\chapter{Conclusiones}

Este proyecto ha supuesto para mi una experiencia nueva, puesto que no se
trata de un desarrollo propio, sino que extender algo ya existente, para
adaptarlo a unas necesidades nuevas. Además he presenciado cómo se ha
creado, y cómo se está creando una federación de identidad desde dentro, a
todos los niveles, desde el nivel técnico, al nivel institucional.

Al principio puede parecer un proyecto muy complejo, puesto que no se
conoce el código del servidor SSH, y realizar una adaptación de algo tan
grande puede resultar abrumador. Gracias a la claridad del código, y a la
separación del mismo, he conseguido realizar la modificación sin demasiadas
complicaciones.

Puedo concluir de esta experiencia, que las licencias libres, ayudan a la
innovación y aprendizaje, ya que te puedes basar en el código y
conocimiento de otras personas, para mejorar y realizar tus proyectos.

Hoy en día, cualquier organización maneja una gran cantidad de máquinas, y
la mayoría se manejarán de manera remota. Para el acceso remoto lo más
rápido y sencillo es utilizar el protocolo SSH, ya que proporciona
seguridad, y flexibilidad.

Por otra parte, cada vez más se está imponiendo la federación de identidad,
tanto en el ámbito universitario, como en el empresarial, por lo tanto la
federación de identidad es una tecnología a tener en cuenta para el futuro
próximo.

Este proyecto hace que la gestión de máquinas de manera remota se
simplifique, al utilizar los conceptos de la federación de identidad en el
acceso por SSH, por lo tanto simplifica la gestión de multitud de máquinas,
de igual manera que la federación de identidad simplifica la gestión de los
usuarios a las aplicaciones web.

Con esta idea de gestión de identidad para el acceso SSH, el permitir o
prohibir el acceso a un servicio SSH es tan simple como visitar una página
web.

Además este proyecto abre la puerta para la federación de identidad en
aplicaciones que no sean web. Este proyecto es una prueba de que es posible
utilizar los mecanismos de federación de identidad para autenticar a
usuarios en una aplicación que utiliza un navegador como medio.

%TODO el resultado de este proyecto puede resultar una aportación
%relevante al uso de la federación.
Por todas estas razones pienso que este proyecto puede ser un comienzo, o
algo importante en esto de la federación de identidad, que a día de hoy se
está imponiendo, puesto que cada vez hay más usuarios, y esto simplifica la
gestión de la identidad de los mismos.


\chapter{Anexos}
    \section{Descripción enviada a un grupo de trabajo}

    Tratando de que el proyecto se conozca fuera de la federación andaluza,
    se ha redactado una pequeña descripción del proyecto en inglés, para
    enviarla a un grupo de trabajo, que trata sobre estos temas.

\begin{quote}

Using Federation credentials to SSH login

This project is based in the paper of Feide:
\url{http://rnd.feide.no/content/feide-and-ssh-secure-shell}

Problem:

    When we have lots of SSH accounts, we need to remember lots of
    passwords and it is in all SSH servers.

    Because that, if we want to change the password, it is needed to
    change it in all servers.  Furthermore, if one of that SSH servers
    is hacked, our password could be known by other person.

    Other problem related to password and SSH could be when you want
    to give a SSH service to many users, and you don't manage that
    users accounts. 

    Identity federation solve that problems, but it is designer only
    for web. This document describe how we use identity federation
    credentials to solve that problems in SSH services.

Objective:

    The objective of this project is use identity federation
    credentials to authenticate in SSH. We try to make easiest for
    user and for the administer of the service.

    We want the user could access to the service without write his own
    password. And could access to others federated servers
    automatically.

Solution:

    To make this, we use the ssh public key system that could be used
    with SSH servers. With that, there is not needed to write the
    password.

    The SSH server need to get the user public key from the federation
    system, but it work only in web applications. For that we put a
    global LDAP server. In that LDAP server we put an entry when the
    user login in the federation, and then from the SSH server we can
    look up if the user is authenticated.

    Our solution has two parts:
    
    1.- SP application

    One is an SP application made in php.  That is a global
    application, only one it's needed for all entities. The user enter
    in that web application, and it's redirected to his own IdP. Then
    when the user is authenticated, it's redirected to that
    application.

    That application receive the user name and ssh public
    key from the IdP, and write an entry in the global LDAP server.

    Furthermore in that application it is possible to write your
    ssh public key manually, if your IdP don't have it, or if your
    aren't in your own machine.

    The application make an entry in the LDAP server with the name of
    the user, concatenated with his entity name, some like
    username-entity (danigm-us), with the ssh public key, and with an
    timeout number.

    Since the login in that web application, the user can login in all
    the SSH servers automatically, without write any password.

    The look of that application is like that [1]

    2.- Openssh path
    
    The ssh server work this way:
        First try to authenticate the user with ssh public key,
        looking for keys in \$HOME/.ssh/authorized\_keys. If that works,
        login without ask for password.

        Second try to authenticate with PAM modules if that it's
        activated.

    At first, we think in a PAM module, but because of the ssh
    working, that method require that the user write a password or
    something similar.

    The way that we take is to modify the openssh server, that it's
    free software.

    To make that we write a patch to that ssh server. The objective of
    that patch is to touch only the needed files to make it strong to
    the variation of the openssh code.

    This patch touch the authenticate method of the server,
    introducing a subroutine exact after the server look for the
    public key in the local filesystem. 

    If the server don't find the user public key, that subroutine it's
    called. That asks for the user to the LDAP server, that we mentioned
    before. If the user is authenticated, the patched ssh server
    receive the public key, and the timestamp, that indicate the
    timeout of the login.

    The subroutine check if the time is correct, and then create a
    temp file in the local filesystem with that rsa public key. After
    that, check again the user identity, but now with the new temp
    file. If all is well, the user is authenticated directly. Else,
    the ssh server authentication method continue asking for the user
    password.

    All the parameters needed by that patch could be sets in the
    sshd.conf file.

    Now we use like public ssh key server a LDAP server, but the patch
    it's implemented to support others servers easily. You only need
    to write a function that ask for the user to the server, and
    return the ssh public key if it exist.

Problems:
    
    One problem for that solution is that the ssh server needs that
    the user has an account in that machine.

    For this problem we write a simple solution, it's a web
    application that is protected by an SP, and when the user login in
    the application, he can create an account in all machines that
    has the patched ssh server, and looks like this [2].

    Other problem is that the user could change the password, and if
    the user change the password, or write a file in the directory
    .ssh, he can login after without authenticate in the federation
    system, and we don't want that.

All the code could be found in the RedIRIS forge [3].

Images:
    [1] sshApp.png
        \begin{center}
            \includegraphics[width=\textwidth]{img/sshApp.png}
        \end{center}
    [2] userAdd.png
        \begin{center}
            \includegraphics[width=\textwidth]{img/userAdd.png}
        \end{center}
Links:
    [3] \url{https://forja.rediris.es/projects/aupaai/}

    \end{quote}

\section{Sobre el despliegue de la aplicación}


    Para una posible implantación del proyecto en un proyecto de RedIRIS,
    se redactó una pequeña guia que facilita el despliegue del mismo.

    \begin{quote}
El despliegue completo consta de tres partes:

1.- Aplicación web php. Solo hace falta desplegarla en una máquina, que
esté protegida por un SP, un SSO o algo así. Esta aplicación escribirá
en el LDAP cada vez que un usuario se autentique.

Hay que configurar las siguientes variables en el fichero
ssh\_backend.php:

//base para la busqueda e introduccion de usuarios logueados en el LDAP
\$base\_dn ='o=People,dc=us,dc=es';
\$servidor\_ldap = "goonie.us.es";
\$puerto\_ldap = 389;
\$bn = 'cn=admin,dc=us,dc=es';
\$pw = 'xxxx';

//tiempo de sesion valido
\$minutes\_timeout = 30;

//El atributo de shibboleth que mira para coger el certificado
\$shib\_header = "HTTP\_USERCERTIFICATE";

//atributo del LDAP para almacenar la clave
\$rsa\_server\_key\_attr = 'sshpublickey';

//atributo del LDAP para el timeout, se guarda como un timestamp
\$rsa\_server\_timeout = 'schacuserstatus';

De momento está pensada para mirar los atributos de shibboleth, pero si
no los recibe, creo que también funciona, solo que el certificado hay
que meterlo a mano.

Mira el nombre de usuario de la variable \$\_SERVER["REMOTE\_USER"] y
debería ser un nombre@dominio.loquesea

2.- LDAP para las claves públicas de los usuarios que se han
autenticado:

Un simple openldap bastaría. Hay que añadirle los esquemas
openssh-lpk\_openldap, para el campo "sshpublickey" y el schac para el
campo "schacuserstatus", que lleva el timeout.

En este LDAP es dónde va a escribir la aplicación php, y de donde van a
leer los diferentes servidores ssh parcheados.

3.- Servidor ssh parcheado:

Lo mando como adjunto, y solo haría falta compilarlo y cambiar la
configuración del fichero sshd\_config

Si se quiere compilar sobre el mismo directorio, si ya hay algún otro
servidor ssh (lo más normal), se recomienda así.
./configure --prefix=\$PWD \&\& make
y se lanzaría:
\$PWD/sshd -f sshd\_config

Si se quiere compilar para el sistema:
./configure \&\& make \&\& make install
y se lanzaría según el sistema.

Configuracion: el fichero sshd\_config
Para la federación se le han añadido una serie de opciones adicionales:

\# Opcion para activar el acceso federado
usefed yes

\# Servidor de claves RSA, el LDAP
fedserver goonie.us.es
fedserver\_root\_dn "cn=admin,dc=us,dc=es"
fedserver\_root\_pw xxxxxx
fedport 389

\# base de busqueda para usuarios en el LDAP
fedserver\_base "o=People,dc=us,dc=es"

\# Atributos de clave, y de timeout
fedserver\_attr sshPublicKey
fedserver\_timeattr schacUserStatus

Si hay otro servidor corriendo, también se puede cambiar el puerto con
Port 2222

-----

Con esto ya habría una versión funcional. Ahora hay que tener unas
consideraciones previas.

1.- Creación de cuentas:
Para que un usuario pueda acceder, además de autenticarse tiene que
tener una cuenta creada en el sistema. Para eso hay otra aplicación,
useradd.php, desde la cual podría hacerse de manera centralizada. Es
decir, esta aplicación muestra una lista de los servidores ssh
disponibles, y si tienes cuenta o no, además se puede solicitar una
cuenta, y se crearía de manera inmediata.

La instalación sería igual que la de ssh, y habría que configurar los
servidores en el index.php:

Por cada servidor una entrada en el array, direccion del servidor, y
descripcion:
\$servers[0] = 'federacion21.us.es';
\$desc[0] = "servidor1";

Para que esta aplicación funcione bien, el servidor donde esté debe
tener acceso por ssh a todos los servidores con cuenta de root. Para
ello hay que añadir en el \$HOME/.ssh/authorized\_keys de cada servidor la
clave pública de este servidor (ssh-keygen, \$HOME/.ssh/id\_rsa.pub) donde
está la aplicación.
Además para que tenga acceso la aplicación, si se está ejecutando con el
usuario www-data, este usuario debería tener acceso a sudo para el
comando ssh.

\#visudo
www-data ALL=NOPASSWD: /usr/bin/ssh

----

Detalles a tener en cuenta:

Se debería bloquear la posibilidad del cambio de password de los
usuarios, ya que si un usuario ejecuta passwd, podrá entrar en un futuro
sin loguearse.
También sería necesario impedir que los usuarios escriban en el
directorio \$HOME/.ssh/ para evitar que metan alguna clave rsa, y poder
tener acceso sin pasar por la federación.

Cómo solucionar esto:
Para bloquear el cambio de password, basta con chmod
554 /usr/bin/passwd.
Para impedir la escritura se puede poner una tara cron que elimine de
cada usuario ese directorio cada cierto tiempo, o se le puede quitar el
acceso al mismo con chmod 000 .ssh/ para cada usuario


Nada más. Pienso que el despliegue, aunque pueda parecer complejo no
debería serlo tanto, por lo menos en debian y en ubuntu a mí no me ha
costado casi nada.


    \end{quote}

%TODO faltan los enlaces a los sitios web, LDAP, Openssh, etc

\newpage


\begin{thebibliography}{widest-label}

 \bibitem[OpenSSH]{openssh} \url{http://openssh.org/}

 \bibitem[Federación Noruega]{feide} \url{http://rnd.feide.no/}
 
 \bibitem[Openssh-lpk]{lpk} \url{http://dev.inversepath.com/trac/openssh-lpk}
 
 \bibitem[PAPI]{papi} \url{http://papi.rediris.es}

 \bibitem[WAYF]{wayf} \url{http://www.switch.ch/aai/support/tools/wayf.html}


\end{thebibliography}


\end{document}
