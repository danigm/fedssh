\chapter{Análisis del problema}
    \section{Necesidad de federación de identidad}
        \subsection{¿Qué es la federación de identidad?}

    La gestión de identidad centralizada fue creada para ayudar con el
    trato de usuarios y seguridad de datos, cuando el usuario accede
    desde dentro de la misma red, o al menos desde dentro del mismo
    dominio de control. Cada vez más, los usuarios están accediendo a
    sistemas externos, que están fuera de su dominio de control, y
    usuarios externos están accediendo a sistemas internos.

    La federación de identidad es un concepto nuevo, nacido a partir
    del uso masivo de la web, así como de la necesidad de
    interoperabilidad entre diferentes entidades que gestionan
    diferentes grupos de usuarios de maneras totalmente diferente.

    Mediante soluciones de Identidad Federada los individuos pueden
    emplear la misma identificación personal (típicamente usuario y
    contraseña) para identificarse en redes de diferentes
    departamentos, empresas o entidades. De este modo las empresas o
    entidades comparten información sin compartir tecnologías de
    directorio, seguridad y autenticación, como requieren otras
    soluciones (metadirectorio, Single Sign On, etc.). Para su
    funcionamiento es necesaria la utilización de estándares que
    definan mecanismos que permiten a las empresas compartir
    información entre dominios. El modelo es aplicable a un grupo de
    empresas o a una gran empresa con numerosas delegaciones y se basa
    en el "círculo de confianza" de estas, un concepto que identifica
    que un determinado usuario es conocido en una comunidad
    determinada y tiene acceso a servicios específicos.

    La federación de identidad, describe las tecnologías, estándares y
    casos de uso los cuales sirven para habilitar la portabilidad de
    información de identidad entre diferentes dominios de seguridad
    autónomos. El objetivo final de la federación de identidad es dar
    acceso seguro a datos, sistemas u otros dominios, a los usuarios
    de un dominio, y sin necesidad de redundancia de administración de
    usuarios.

    El uso de la federación de identidad puede reducir costes
    eliminando la necesidad de escalar el sistema de gestión de
    identidad. La federación de identidad puede incrementar la
    seguridad y reducir los riesgos habilitando a una organización a
    identificar y autenticar un usuario una vez, y luego usar esta
    información de identidad entre diferentes sistemas, incluso entre
    páginas de organizaciones externas. Puede mejorar la privacidad
    permitiendo al usuario gestionar qué información es compartida, o
    limitando la cantidad de información compartida. Además puede
    mejorar drásticamente la experiencia del usuario final eliminando
    la necesidad de registrar una cuenta nueva, o la necesidad de
    tener que hacer login de manera redundante.

    Empresas líderes en todo el mundo han desplegado la federación de
    identidad para acercarse a sus socios, mejorar el servicio de
    compra, acelerar la ejecución de negocios, recortar costes y
    complejidad de integración con otros servicios, y liberarse a sí
    mismos de la autenticación.

        \subsection{Partes en la federación de identidad}

    Un sistema de federación de identidad está formado por diferentes
    partes, o sistemas necesarios para dar el servicio. A continuación
    se detallan las partes más importantes de este tipo de sistema,
    junto con algunos ejemplos de posibles tecnologías que pueden
    desempeñar el papel.

            \begin{itemize}

            \item \textbf{Single Sing On (SSO)}

    Un sistema Single Sing On, es un sistema que permite simplificar
    el acceso por parte del usuario, y de las aplicaciones, a
    diferentes aplicaciones, que comparten los mismos usuarios, y por
    tanto el mismo sistema de autenticación.

    Este sistema nace de la necesidad de autenticar y autorizar al
    mismo grupo de usuarios, pero en diferentes aplicaciones. Por lo
    tanto se propone un sistema de autenticación único, de tal modo
    que todas las aplicaciones utilicen este mismo sistema, y
    mejorando así la experiencia del usuario final, ya que siempre
    introducirá sus credenciales en un sistema homogéneo,
    independientemente de la aplicación, y además la sesión permanece
    entre diferentes aplicaciones. Por parte de las aplicaciones que
    utilizan este sistema, evita la necesidad de tratar con la
    identidad de estos usuarios, ya que el sistema de Single Sing On
    es el que los autentica.

    SSO es un método de control de acceso que permite autenticar al
    usuario una vez, y conseguir acceso a diferentes recursos en
    diferentes sistemas software.

    En una infraestructura homogénea, o al menos en una en la que
    exista un único esquema de autenticación, o donde existe una base
    de datos centralizada de usuarios, el sistema SSO es una gran
    ventaja.

    En la federación de identidad, el Sigle Sing On no es algo
    necesario, pero sí recomendable. No es un requisito indispensable
    para montar un sistema de gestión de identidad federada, pero sí
    que facilita mucho el acceso por parte del usuario.

    Por lo tanto en la mayoría de los sistemas de federación de
    identidad se hace uso de SSO.

    Algunos ejemplos de software que puede servir como SSO:
    \begin{itemize}
        \item \textbf{OpenID} \href{http://openid.net/}http://openid.net/
        \item \textbf{PAPI} \href{http://papi.rediris.es}http://papi.rediris.es
        \item \textbf{Sun Access Manager}
        \href{http://www.sun.com/software/products/access_mgr/index.jsp}http://www.sun.com/software/products/access\_mgr/index.jsp
        \item \textbf{OpenSSO} \href{https://opensso.dev.java.net/}https://opensso.dev.java.net/
    \end{itemize}

            \item \textbf{Proveedor de identidad (IdP)}

    Un proveedor de identidad es un sistema que gestiona la identidad
    de los usuarios de una organización, y ofrece un servicio de
    autenticación a otras aplicaciones.

    En la práctica un Proveedor de Identidad es una aplicación web a
    la cual serán redirigidos todos los usuarios de una organización
    que quieran acceder a una aplicación federada, donde se
    autenticarán, y este IdP ofrecerá los datos de autenticación a las
    diferentes aplicaciones federadas.

    Es necesario que haya un IdP por cada organización participante en
    la federación de identidad, puesto que es la vía de autenticación
    de sus usuarios internos.

    Es aquí donde se hace importante el sistema SSO del que hemos
    hablado antes, ya que se puede enlazar el IdP con el sistema de
    SSO para ofrecer un sistema de autenticación único para todas las
    aplicaciones federadas, ya sean internas a la organización o
    externas, por lo que el usuario final tendrá un acceso homogéneo a
    todo el sistema federado.

    Algunos ejemplos de software que pueden dar el servicio de Proveedor
    de Identidad:
    \begin{itemize}
        \item \textbf{Shibboleth IdP} \href{http://shibboleth.internet2.edu/}http://shibboleth.internet2.edu/
        \item \textbf{SimpleSAMLPHP} \href{http://code.google.com/p/simplesamlphp/}http://code.google.com/p/simplesamlphp/
    \end{itemize}

            \item \textbf{Proveedor de servicio (SP)}

    Un proveedor de servicio es la otra parte de la federación. Todas
    las aplicaciones web federadas, deberán estar protegidas por un
    proveedor de servicio.

    El SP permite impedir el acceso a las aplicaciones, redirigiendo a
    los usuarios a sus correspondientes IdP, para que se autentiquen,
    y en ese caso permite el acceso, comunicándose con el IdP, que le
    proporcionará la información necesaria para la autorización, y el
    correcto funcionamiento de las aplicaciones.

    Un SP está estrechamente relacionado con un servidor web, y se
    comunica con los diferentes IdPs de la federación. Esta es la
    forma que tiene el sistema de pasar los datos de identidad de una
    organización a otra, de forma segura, a través del protocolo SAML.

    Cada aplicación web federada debería estar protegida por un SP,
    así pues puede haber más de un SP por cada organización.

    Dada la estrecha relación entre el IdP y el SP, el software
    necesario para dar este servicio se puede encontrar en las mismas
    páginas. Tanto Shibboleth, como simplesamlphp ofrecen tanto el
    servicio de IdP, como el de SP:

    \begin{itemize}
        \item \textbf{Shibboleth IdP} \href{http://shibboleth.internet2.edu/}http://shibboleth.internet2.edu/
        \item \textbf{SimpleSAMLPHP} \href{http://code.google.com/p/simplesamlphp/}http://code.google.com/p/simplesamlphp/
    \end{itemize}

    En teoría es independiente el software que se utilice como IdP, o
    como SP por las diferentes organizaciones, ya que todos deberían
    compartir el mismo protocolo, SAML.

            \item \textbf{Where are you from (WAYF)}

    El objetivo del servicio "Where are you from" (WAYF) es guiar al
    usuario a su propio Proveedor de Identidad (IdP). A veces es
    llamado "Identity Provider Discovery".

    Básicamente lo que hace es presentar al usuario una lista de
    Proveedores de Identidad y redirigir al navegador del usuario al
    IdP seleccionado.

    Algunos sistemas WAYF implementan otras funcionalidades
    adicionales, que mejoran la experiencia del usuario, y la
    facilidad de uso. Por ejemplo existen varios métodos para recordar
    o adivinar el IdP del usuario.

    Un ejemplo de software que implementa este servicio se puede
    encontrar:
    
    \begin{itemize}
        \item \textbf{Switch} \href{http://www.switch.ch/aai/support/tools/wayf.html}http://www.switch.ch/aai/support/tools/wayf.html
    \end{itemize}

    El WAYF es una parte importante en el sistema de la federación,
    puesto que es el elemento que conecta los Proveedores de
    Servicio (SP) con los Proveedores de Identidad (IdP). Cuando un
    usuario intenta acceder a una aplicación protegida tras un SP,
    este será redirigido al WAYF, donde el usuario seleccionará cuál
    es su organización, y en consecuencia el sistema WAYF redirigirá
    al usuario al Proveedor de Identidad pertinente.

            \item \textbf{Servicio de directorio (LDAP)}

    ``Un servicio de directorio (SD) es una aplicación o un conjunto
    de aplicaciones que almacena y organiza la información sobre los
    usuarios de una red de ordenadores, sobre recursos de red, y
    permite a los administradores gestionar el acceso de usuarios a
    los recursos sobre dicha red. Además, los servicios de directorio
    actúan como una capa de abstracción entre los usuarios y los
    recursos compartidos.''

    Realmente no es imprescindible el servicio de directorio para
    desplegar una arquitectura de identidad federada, pero normalmente
    siempre va ligada a este tipo de aplicaciones. Esta estrecha
    relación se debe a que últimamente se está imponiendo el uso de
    esta tecnología para la gestión de identidad, de manera local. Por
    lo tanto la gran mayoría de las organizaciones ya tendrán un
    servicio de directorio para gestionar sus usuarios.

    La mayoría de sistemas y aplicaciones para desplegar una
    federación de identidad, están desarrollados pensando en este
    hecho, por lo tanto facilita la tarea.

    Sin embargo, es posible utilizar bases de datos, y otros tipos de
    almacenamiento o gestión de usuarios, aunque la configuración será
    más compleja, y tal vez no sea tan óptima.


            \end{itemize}


    \section{Uso de la federación fuera de la web}

    \section{La importancia del ssh}
         (túneles, importación X ...)
