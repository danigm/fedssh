\chapter{Despliegue de una maqueta en CONFIA}

    Para la prueba del proyecto, se ha utilizado la federación CONFIA
    (\url{http://confia.aupa.info/}),
    montada en fase de pruebas. CONFIA es la federación de identidad de las
    universidades andaluzas, es un proyecto pionero en España, que está
    comenzando ahora. Su objetivo principal es montar una infraestructura
    de identidad federada a nivel Andaluz, con todas las universidades
    andaluzas.

    Esta federación no está aún en producción, y utiliza usuarios de
    prueba, pero es prácticamente funcional. Actualmente utiliza el
    protocolo Shibboleth 1.3, y hay varios IdPs (Universidad de Sevilla,
    Universidad de Málaga, Universidad de Córdoba, etc), y varios SPs
    protegiendo diferentes aplicaciones en diferentes universidades.

    Además existe una infraestructura de metadatos y un WAYF global,
    que están en el CICA.

    Así pues, el proyecto se puede desplegar en este entorno, y en este
    entorno es en el que se ha desarrollado, sobretodo en la parte de la
    Universidad de Sevilla.

    Por parte de la Universidad de Sevilla, hay un IdP de Shibboleth, sobre
    un servidor Tomcat, que utiliza como sistema de SSO el SUN Access
    Manager.


    En la máquina \texttt{federacion21.us.es} hay un servidor Apache2
    protegido tras un SP de shibboleth.

\section{Máquinas involucradas}

    Hay varios servidores SSH parcheados funcionando dentro de la
    federación, a modo de ejemplo. Los servidores SSH parcheados se han
    colocado en el puerto 2222, para no interferir con los servidores SSH
    por defecto de cada máquina.

    \begin{itemize}

    \item \texttt{federacion21.us.es}, sistema operativo Ubuntu.
    \item \texttt{federacion22.us.es}, sistema operativo Ubuntu.
    \item \texttt{goonie.us.es}, sistema operativo debian. Máquina en la
    cuál se ha desarrollado todo el proyecto.

    \end{itemize}

    Tanto federacion21 como federacion22, son máquinas virtuales que corren
    sobre un entorno de virtualización XEN. Así pues el despliegue de
    pruebas no pone en peligro ninguna infraestructura delicada, puesto que
    estos servidores no contienen información sensible, ni prestan ningún
    servicio.

    Las aplicaciones web necesarias para la autenticación, y para la
    creación de cuentas están desplegadas en el servidor
    \texttt{federacion21.us.es}, tras el servidor web apache2, y protegidas
    con un SP.

    El servidor de claves es un servidor openldap, instalado en
    \texttt{goonie.us.es}, que almacena las claves públicas, los nombres de
    usuarios, y el tiempo de sesión, que luego consultan los servidores SSH
    parcheados.

    Dada esta infraestructura de pruebas implementada, hoy en día es posible
    acceder a estos servidores SSH parcheados con cualquier usuario de la
    Universidad de Málaga, que se autentique en la federación, o de la
    Universidad de Córdoba, etc.

\section{Servicio SSH federado en la Universidad de Sevilla}

    %TODO llevar esto a los agradecimientos
    La Universidad de Sevilla es una de las impulsoras del proyecto, siendo
    parte importante en él, y cobrando relevancia en las pruebas
    realizadas.

    A este proyecto se le ha facilitado acceso a todas las máquinas
    necesarias, y se le ha dado permiso para implantar un entorno de
    prueba, permitiendo así probar la viabilidad del sistema en un entorno
    más o menos real, y con diferentes organizaciones implicadas.

    De no ser así, habría sido imposible probar este proyecto, puesto que
    la infraestructura que hay detrás del proyecto implica tener muchas
    máquinas, en localizaciones diferentes, y con una base de usuarios de
    prueba.

    Además de los medios materiales, me han permitido el asistir a
    diferentes reuniones de trabajo del proyecto CONFIA, lo que me ha
    permitido compartir opiniones y difundir este proyecto, más allá del
    ámbito puramente académico.

\section{Pruebas}

    Se han realizado pruebas de acceso a diferentes servidores SSH, (tanto a
    federacion21, como a federacion22), con usuarios de la federación.

    En principio las pruebas se han realizado con usuarios de la misma
    Universidad de Sevilla. Puesto que el proveedor de identidad de la
    Universidad trabaja con una maqueta que aprovisiona los usuarios, no se
    trata de usuarios reales, en producción, sino que son ficticios, aunque
    basados en datos reales.

    Por otra parte, una vez el proyecto se consideró lo suficientemente
    maduro, se publicó el sistema a toda la federación, y diferentes
    universidades, desde sus propias instalaciones han probado y verificado
    el correcto funcionamiento del proyecto.

    Así pues se puede decir que se han realizado varias pruebas, con
    usuarios ficticios, y reales, aunque aún querían una serie de pruebas
    más estrictas, con mayor número de usuarios, y un nivel de concurrencia
    alto, para comprobar qué carga soporta, y comprobar que no ocurre
    ningún fallo no previsto.

\section{Futuros despliegues}

    \begin{itemize}

    \item El proyecto se ha desarrollado en paralelo con la federación de
    identidad de las universidades andaluzas, y en colaboración con CONFIA,
    por lo que es posible que cuando la federación se implante, alguna
    Universidad quiera dar este servicio. Por esa parte el proyecto tiene
    futuro dentro de las universidades andaluzas.

    %TODO explicar que el proyecto es interesante, que se ha
    %presentado en foros europeos, y es posible que se despliegue en
    %otras federaciones
    \item Al partir de un articulo publicado por la federación noruega, feide, es
    posible que esta esté interesada en probar el proyecto en su propia
    infraestructura, o quizás coger el código y adaptarlo a sus
    necesidades, mejorarlo, o realizar otro proyecto a partir de este. En
    todo caso espero que este proyecto sirva para otras federaciones.

    \item El supercomputador más potente de España, llamado "MareNostrum",
    está situado en Barcelona. Actualmente se está dando acceso a
    diferentes universidades a partes de este supercomputador, para
    aprovechar su capacidad de cálculo en la investigación universitaria.
    Este caso es para el que se ha desarrollado este proyecto.

    \item Además de para federaciones, este proyecto se puede utilizar en
    organizaciones o empresas que no pertenezcan a ninguna federación.
    Sería tan fácil como modificar el código de la aplicación php, y hacer
    que consiguiera los datos de otro lugar, ya sea una base de datos, un
    servicio de directorio, etc. Por esto podría ser interesante para
    cualquier empresa que tenga que lidiar con una gran cantidad de
    servidores, el no tener que cambiar contraseñas cada poco tiempo, ni
    tener que crear usuarios nuevos en cada máquina.

    \end{itemize}
