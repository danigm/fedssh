\chapter{Despliegue de una maqueta en CONFIA (Federación de Identidad de las Universidades Andaluzas)}

    Para la prueba del proyecto, se ha utilizado la federación CONFIA
    (\url{http://confia.aupa.info/}),
    montada en fase de pruebas. CONFIA es la federación de identidad de las
    Universidades Andaluzas, es un proyecto pionero en España, que está
    comenzando ahora. Su objetivo principal es montar una infraestructura
    de identidad federada a nivel Andaluz, con todas las universidades
    Andaluzas.

    Esta federación no está aún en producción, y utiliza usuarios de
    prueba, pero es prácticamente funcional. Actualmente utiliza el
    protocolo Shibboleth 1.3, y hay varios IdPs (Universida de Sevilla,
    Universidad de Málaga, Universidad de Córdoba, etc), y varios SPs
    protegiendo diferentes aplicaciones en diferentes universidades.

    Además existe una infraestructura de metadatos y un WAYF global, que está
    en el CICA.

    Así pues, el proyecto se puede desplegar en este entorno, y en este
    entorno es en el que se ha desarrollado, sobretodo en la parte de la
    Universidad de Sevilla.

    Por parte de la Universidad de Sevilla, hay un IdP de Shibboleth, sobre
    un servidor Tomcat, que utiliza como sistema de SSO el SUN Access
    Manager.


    En la máquina \texttt{federacion21.us.es} hay un servidor Apache2
    protegido tras un SP de shibboleth.

    \section{Máquinas involucradas}

    Hay varios servidores SSH parcheados funcionando dentro de la
    federación, a modo de ejemplo. Los servidores SSH parcheados se han
    colocado en el puerto 2222, para no interferir con los servidores SSH
    por defecto de cada máquina.

    \begin{itemize}

    \item \texttt{federacion21.us.es}, sistema operativo Ubuntu
    \item \texttt{federacion22.us.es}, sistema operativo Ubuntu
    \item \texttt{goonie.us.es}, sistema operativo debian. Máquina en la
    cuál se ha desarrollado todo el proyecto.

    \end{itemize}

    Las aplicaciones web necesarias para la autenticación, y para la
    creación de cuentas están desplegadas en el servidor
    \texttt{federacion21.us.es}, tras el servidor web apache2, y protegidas
    con un SP.

    El servidor de claves es un servidor openldap, instalado en
    \texttt{goonie.us.es}, que almacena las claves públicas, los nombres de
    usuarios, y el tiempo de sesión, que luego consultan los servidores SSH
    parcheados.

    Dada esta infraestructura de pruebas implementada, hoy en día es
    posible acceder a estos servidores SSH parcheados con cualquier usuario
    que se autentique en la federación. Es posible acceder con cualquier
    usuario de la Universidad de Málaga, que se autentique en la
    federación, o de la Universidad de Córdoba, etc.

    \section{Servicio prestado, por parte de la Universidad de Sevilla}
    \section{Pruebas}
    \section{Futuros despliegues}
