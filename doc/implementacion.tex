
\chapter{Implementación y despliegue}
    \label{implementacion}
    \section{Cambios al openssh}

    El código del parche para el servidor SSH debe tocar lo mínimo
    posible, para tener seguridad a la hora de aplicar actualizaciones
    de openssh. Así pues, la idea principal y que ha guiado el
    desarrollo de esta modificación ha sido esta, y por lo tanto se ha
    creado un parche lo más simple posible, con las menores
    dependencias de librerías externas, y fácilmente ampliable,
    pensando en casos futuros.

    Lo primero que hay que hacer para hacer una mejora o una adaptación de
    un proyecto software libre, es descargarse el código, y empezar a
    estudiarlo, para ver en qué partes del mismo hay que encajar la nueva
    funcionalidad.

    Para descargar el código se ha usado el sistema de control de versiones
    que utilizan para este proyecto, CVS
    (\url{http://www.openssh.com/portable.html}), bajando la versión para
    linux, puesto que es el sistema sobre el cuál se ha desarrollado todo
    el proyecto.
    
    \begin{verbatim}

    export CVSROOT=anoncvs@anoncvs.mindrot.org:/cvs
    export CVS_RSH=/usr/bin/ssh
    cvs get openssh

    \end{verbatim}

    Una vez conseguido el código, hubo un proceso de estudio del mismo,
    para entender minimamente el funcionamiento de la aplicación, antes de
    empezar a tocar nada.

    Tras estudiar el código, y realizar varias pruebas, se localizó la zona
    de código dónde se realiza la autenticación del usuario, y se comienza
    a modificar este código para añadirle la funcionalidad que estamos
    buscando.

    El fichero fuente, dentro de openssh, que se encarga de la
    autenticación SSH es el \texttt{auth2-pubkey.c}. Concretamente la
    autenticación por clave pública se realiza en la función
    \texttt{int user\_key\_allowed(struct passwd *pw, Key *key)}, por lo
    tanto esta es la función que hay que modificar.

    Concretamente este trozo de código es el encargado de comprobar que
    en el servidor existe un fichero que contiene las claves públicas que
    tienen acceso.

    \begin{lstlisting}

    file = authorized_keys_file(pw);
    success = user_key_allowed2(pw, key, file);
    xfree(file);
    if (success)
        return success;

    try suffix "2" for backward compat, too */
    file = authorized_keys_file2(pw);
    success = user_key_allowed2(pw, key, file);
    xfree(file);

    return success;

    \end{lstlisting}

    La variable \textit{file} tiene la ruta hacia el fichero dónde están las claves
    públicas para acceder, así pues añadir a este código la idea de que si
    un usuario no puede acceder por el método estandar, se compruebe si
    está en la federación, es tan simple como:


    \begin{lstlisting}

    file = authorized_keys_file(pw);
    success = user_key_allowed2(pw, key, file);
    xfree(file);
    if (success)
        return success;

    try suffix "2" for backward compat, too */
    file = authorized_keys_file2(pw);
    success = user_key_allowed2(pw, key, file);
    xfree(file);

    if (success)
        return success;

    // try external file fed+ssh <danigm>
    if(options.usefed == 1){
        get_rsa_key_ldap(options.fedserver, options.fedport,
                pw->pw_name, rsa_key);
        debug("RSA_EXTERNAL_KEY: trying this -> %s\n",rsa_key);

        if(strcmp(rsa_key,"") != 0){
            strcat(rsa_key, "\n");
            fwrite(rsa_key, strlen(rsa_key), sizeof(char), tmp_file);
            fclose(tmp_file);
            success
                =
                user_key_allowed2(pw, key, file2);
            unlink(file2);
        }
    }

    return success;

    \end{lstlisting}

    De esta manera, si no se ha podido acceder de la manera convencional,
    si está puesta la opción de usar el acceso federado, se hace una
    llamada a la función \texttt{get\_rsa\_key\_ldap}, que devolverá una
    cadena con la clave pública de este usuario si está autenticado, y en
    tal caso se creará un fichero temporal que se usará para intentar
    autenticar.

    Este simple cambio, y alguno más para definir las posibles opciones
    configurables con el fichero de configuración, es lo que se cambiaría
    de verdad sobre el código del servidor ssh real, por lo que el riesgo
    de introducir fallos adicionales al código se reduce considerablemente.

    Por tanto se delega la responsabilidad de saber si el usuario está
    autenticado o no a la función \texttt{get\_rsa\_key\_ldap}. Esta
    función se puede modificar, y cambiar por cualquier otra que devuelva
    una cadena con la clave pública del usuario que se le está pasando como
    argumento. Por esto es fácil implementar la comunicación con otros
    sistemas que no sean un servicio de directorio, o por alguna función
    que haga alguna operación más.

    Ya que la complejidad de la autenticación se ha delegado en esta
    función, se ha implementado en un fichero aparte, que se introduce en
    las herramientas de compilación, \texttt{automake}, y se enlaza para
    que sea accesible desde el código anteriormente comentado.

    Lo que hace la función básicamente es realizar una consulta al servidor
    LDAP buscando al usuario en cuestión. Si existe este, se extraen los
    datos de clave pública y timeout. Inmediatamente después se comprueba
    que el timeout no esté cumplido, y en ese caso se devuelve la clave
    pública del usuario, y se termina. En caso de que el timeout esté
    cumplido, o que no se encuentre al usuario, se devuelve una cadena
    vacía.

    En realidad la función es bastante sencilla y simple, pues así lo hemos
    requerido, y este era nuestro objetivo. Donde radica la mayor
    complejidad es en el uso de las llamadas a LDAP en el lenguaje de
    programación C, puesto que el acceso no es trivial, y durante el
    desarrollo del proyecto, en la infraestructura de pruebas se ha
    modificado la librería de acceso openldap, cambiando la forma de
    acceso, y poniendo las funciones utilizadas en la primera versión como
    \texttt{deprecated}, por lo tanto se ha tenido que reescribir el código
    para utilizar las nuevas funciones.

    Aquí está el código realmente relevante de esta función:

    \begin{lstlisting}
    
    LDAPMessage *entry=ldap_first_entry(ld, msg);
    for( attr = ldap_first_attribute(ld, entry, &ber);
            attr != NULL; attr = ldap_next_attribute(ld, entry, ber))
    {
        vals = ldap_get_values_len(ld, entry, attr);
        if (vals != NULL)  {
            for(i = 0; vals[i] != NULL; i++) {
                val = vals[i];
                /* process the current value */
                if (strcmp(attr, timeattr) == 0){
                        strcpy(timeout, val->bv_val);
                    }
                if (strcmp(attr, attribute) == 0){
                        strcpy(rsa_key2, val->bv_val);
                        debug("1 %s:%s\n", %attr, %rsa_key2);
                    }
            }
            if (check_timeout(timeout)) {
                    strcpy(rsa_key, rsa_key2);
                    debug("2 %s:%s\n", %attr, %rsa_key);
                }else debug("\nTIMEOUT CUMPLIDO\n");
        }
        ldap_value_free_len(vals);
    }


    \end{lstlisting}

    \section{Aplicaciones federadas: ssh, useradd}

    Como se vió en \ref{login}, la aplicación de login tiene un cometido
    específico, y una funcionalidad claramente definida. Por lo tanto la
    implementación de la misma ha sido el objetivo más fácil dentro de este
    proyecto.

    Se ha elegido la tecnología PHP, por su rápidez para el desarrollo web,
    además de por su simple integración con todo tipo de servidores web, y
    en nuestro caso con apache2. Además, dado que se ha decidido utilizar
    un servidor de claves basado en un servicio de directorio, se ha
    buscado una tecnología que ofrezca simplicidad a la hora de realizar
    estos accesos, y PHP cumple todos los requisitos.

    Se ha descompuesto la aplicación en dos partes diferenciables, para
    seguir un modelo vista controlador. De tal forma que las operaciones
    con datos, y la lógica de la aplicación esté centralizada en
    \texttt{ssh\_backend.php}, y que la parte de muestra de información, y
    formateo de los datos con html esté en \texttt{ssh.php}.

    Explicaré paso a paso las funciones que se pueden encontrar en
    \texttt{ssh\_backend.php}.
    
    \begin{itemize}

    \item En primer lugar hay una serie de variables que definen las opciones del
    entorno dónde queramos desplegar la aplicación.

    \begin{lstlisting}

    $base_dn ='o=People,dc=us,dc=es';
    $servidor_ldap = "goonie.us.es";
    $puerto_ldap = 389;
    $bn = 'cn=admin,dc=us,dc=es';
    $pw = 'xxxxx';
    $minutes_timeout = 30;
    $shib_header = "HTTP_USERCERTIFICATE";
    $rsa_server_key_attr = 'sshpublickey';
    $rsa_server_timeout = 'schacuserstatus';

    \end{lstlisting}

    Están los datos para el acceso al servidor de claves, el tiempo valido
    de sesión, el atributo de shibboleth dónde vendrá el certificado desde
    el IdP, y los atributos en el servidor de claves, dónde se almacenarán
    los datos.


    \item función \texttt{modify}
    
    \begin{lstlisting}

    function modify($ds, $uid, $pubkey){
        global $base_dn;
        global $minutes_timeout;
        global $rsa_server_key_attr;
        global $rsa_server_timeout;
        // preparar los datos
        $timeout = $minutes_timeout * 60; //5 minutos
        $hoy = getdate();
        $timeout = $hoy[0]+$timeout;
        $dn = "uid=".  $uid .",".  $base_dn;
        $info[$rsa_server_key_attr][0] = $pubkey;
        $info[$rsa_server_timeout][0] = "schac:userStatus:us.es:timeout:" . $timeout;

        // anadir la informacion al directorio
        $r=ldap_modify($ds, $dn, $info);
        $h = getdate($timeout);
        
        echo '<p class="info">'._('Esta sesion de ssh es valida 
        hasta: ').$h["hours"].':'.$h["minutes"].':'.$h["seconds"].
        ' - '.$h["mday"].' '.$h["month"].' '.$h["year"].'</p>';

        return $r;
    }

    \end{lstlisting}

    Esta función modifica una entrada ya existente en el directorio,
    cambiando la clave pública y poniendo nuevamente el timeout.

    \item función \texttt{add}

    \begin{lstlisting}

    function add($ds, $uid, $sn, $cn, $pubkey){
            global $base_dn;
            global $minutes_timeout;
        global $rsa_server_key_attr;
        global $rsa_server_timeout;
            // preparar los datos
            $timeout = $minutes_timeout * 60; //5 minutos
            $hoy = getdate();
            $timeout = $hoy[0]+$timeout;
            $dn = "uid=". $uid .",". $base_dn;
            $info["objectClass"][0] = "person";
            $info["objectClass"][1] = "ldapPublicKey";
            $info["objectClass"][2] = "schacUserEntitlements";
            $info["uid"] = $uid;
        if($sn == '')
            $sn = $uid;
        if($cn == '')
            $cn = $uid;
            $info["sn"] = $sn;
            $info["cn"] = $cn;

            $info[$rsa_server_key_attr] = $pubkey;
            $info[$rsa_server_timeout] = "schac:userStatus:us.es:timeout:" . $timeout;

            // anadir la informacion al directorio
            $r=ldap_add($ds, $dn, $info);
        $h = getdate($timeout);
        echo '<p class="info">'._('Esta sesion de ssh es valida
        hasta: ').$h["hours"].':'.$h["minutes"].':'.$h["seconds"].
        ' - '.$h["mday"].' '.$h["month"].' '.$h["year"].'</p>';
            return $r;
    }

    \end{lstlisting}

    Añade una nueva entrada en el servidor de claves.

    \item función \texttt{get\_remote\_user}

    \begin{lstlisting}

    function get_remote_user(){
            $cad = $_SERVER["REMOTE_USER"];

            //$cad = $_SERVER["REMOTE_USER"];
            //partiendo el nombre, para crear danigm-us a partir de danigm@us.es
            $cad = $_SESSION['user'];
            $nado = explode('@', $cad);
            $name = $nado[0];
            $dominio = substr($nado[1], 0, stripos($nado[1], '.'));
            $name = $name.'-'.$dominio;
            return $name;
    }

    \end{lstlisting}

    Devuelve el uid del usuario que se utilizará como nombre de usuario
    para el acceso por SSH. Como cada uid no tiene por qué ser único, pues
    to que en diferentes organizaciones pueden existir usuarios con el
    mismo id, se concatena el dominio del mismo para poder diferenciar a
    usuarios con el mismo uid, pero pertenecientes a diferentes
    organizaciones. Se ha optado por utilizar el caracter '-' como
    separador, puesto que si se utiliza la '@', complicaría un poco el
    acceso por ssh.

    %TODO cuidado que se sale el texto
    \item función \texttt{get\_certificate}
    \begin{lstlisting}
    
    function get_certificate(){
        global $shib_header;
            $certificate = "";
        if (isset($_POST['key'])) {
            // Public key was not received from IdP. First check if user is posting its public key
            $certificate = $_POST['key'];
            }
            // Check if userCertificate attribute is set in SAML response
            else if (isset($_SERVER[$shib_header])) {
                    $certificate = $_SERVER[$shib_header];
                    $certificate = base64_decode($certificate);
            }
            // Trim certificate string
            $certificate = trim($certificate);
            $certificate = str_replace("\r","",$certificate);
            $certificate = str_replace("\n","",$certificate);

            return $certificate;
    }
    \end{lstlisting}

    Intenta conseguir la clave pública del usuario a través de las
    cabeceras que introduce el SP de shibboleth, a través de los datos que
    le manda el IdP.

    \item función \texttt{check\_certificate}
    \begin{lstlisting}

    function check_certificate($certificate){
        // Check if certificate syntax is correct
        if (substr($certificate, 0, 7) == "ssh-rsa" ||
           substr($certificate, 0, 7) == "ssh-dss")
                return true;
        else
                return false;
    }
    \end{lstlisting}

    Comprueba si un certificado es valido.

    \item función \texttt{get\_certificate\_used}
    \begin{lstlisting}

    function get_certificate_used($uid){
        global $rsa_server_key_attr;
        $timestamp = get_attr($uid, $rsa_server_timeout);
        $timestamp = split(":", $timestamp);
        $timestamp = $timestamp[count($timestamp)-1];
        $now = getdate();
        if ($now > $timestamp)
        return get_attr($uid, $rsa_server_key_attr);
        else
        return "";
    }

    \end{lstlisting}

    Mira en el servicio de directorio, para un usuario ya existente cuál es
    la última clave pública que se utilizó.

    \item función \texttt{get\_attr}
    \begin{lstlisting}

    function get_attr($uid, $attr){
            global $base_dn;
            global $servidor_ldap;
            global $puerto_ldap;
            global $bn, $pw;

            //Conectando con el ldap
            $ds=ldap_connect($servidor_ldap, $puerto_ldap)
                or die("No ha sido posible conectarse al servidor ".$servidor_ldap."");
            //Version del protocolo que vamos a usar
            ldap_set_option($ds, LDAP_OPT_PROTOCOL_VERSION, 3);
            //Bind como usuario, vamos a buscar, para ver si ya esta
            ldap_bind($ds, $bn, $pw) or
                die("No ha sido posible enlazar con el servidor".
                $servidor_ldap." con el usuario ".$bn."");

            //se guardan por uid, por lo que filtramos por este campo
            $filter = '(uid='.$uid.')';
            $resource = ldap_search($ds, $base_dn, $filter);
            $info = ldap_get_entries($ds, $resource);
        ldap_unbind($ds);
        if ($info["count"] == 0){
            return null;
        }
        else{
            return $info[0][$attr][0];
        }

    }
    \end{lstlisting}

    Busca en el servicio de directorio, para un usuario existente un
    atributo.

    \item función \texttt{doit}
    \begin{lstlisting}

    /**
     * Esta funcion mira en el servidor de claves, si este usuario esta ya
     * si esta modifica la clave, y el timeout
     * si no esta lo anade
    **/
    function doit($uid, $pubkey){
            global $base_dn;
            global $servidor_ldap;
            global $puerto_ldap;
            global $bn, $pw;

            //Conectando con el ldap
            $ds=ldap_connect($servidor_ldap, $puerto_ldap) or die("No ha sido posible conectarse al servidor ".$servidor_ldap."");
            //Version del protocolo que vamos a usar
            ldap_set_option($ds, LDAP_OPT_PROTOCOL_VERSION, 3);
            //Bind como usuario, vamos a buscar, para ver si ya esta
            ldap_bind($ds, $bn, $pw) or die("No ha sido posible enlazar con el servidor ".$servidor_ldap." con el usuario ".$bn."");

            //se guardan por uid, por lo que filtramos por este campo
            $filter = '(uid='.$uid.')';
            $resource = ldap_search($ds, $base_dn, $filter);
            $info = ldap_get_entries($ds, $resource);

            if ($info["count"] > 0){
                    //solo hay que anadir un pubkey, si es distinto
                    //y modificar el timeout.
                    $response = modify($ds, $uid, $pubkey);
            }else {
                    //nueva entrada
                    $cn = $_SERVER["HTTP_SHIB_PERSON_COMMONNAME"];
                    $sn = $_SERVER["HTTP_SHIB_PERSON_SURNAME"];
                    $response = add($ds, $uid, $sn, $cn, $pubkey);
            }

            ldap_unbind($ds);
            return $response;
    }
    \end{lstlisting}


     Mira en el servidor de claves, si este usuario está ya. Si está
     modifica la clave, y el timeout. Si no está lo añade.

    \item función \texttt{anadir\_usuario}
    \begin{lstlisting}

    /**
     * Codigos de error:
     * -1 certificado no valido
     * -2 No se ha podido completar la operacion
     * -3 El certificado esta en blanco, no lo ha establecido el idp
     *  1 Todo bien 
    **/
    function anadir_usuario(){
            // If $certificate is set, public key has been successfully received
            $certificate = get_certificate();
            if ($certificate != "") {
                    // Check if certificate syntax is correct
                    if(check_certificate($certificate)) {
                            $name = get_remote_user();
                            $response = doit($name, $certificate);
                            // Check if command executed successfully
                            if($response)
                                    return 1;
                            else 
                                    return -2;
                    } else {
                            return -1;
                    }
            }
            else return -3;
    
    \end{lstlisting}

    Esta función intenta conseguir el certificado pasado por el formulario
    de la página, si no lo consigue, intenta conseguirlo mirando los
    atributos pasados por el IdP.

    Una vez que tiene el certificado, comprueba su validez, y si es
    correcto, añade al usuario en el servidor de claves.

    \end{itemize}

    %TODO quitar estos cachos de código
    Por otro lado está la parte que muestra la información,
    \texttt{ssh.php}, cuyo funcionamiento es muy simple, y sólo hace
    llamadas al \texttt{ssh\_backend.php}.

    \begin{lstlisting}

<?php 

session_start();
//soporte para traducciones
$lang=$_GET['lang'];
if($lang=='en')
    $language="en_US.utf8";
else
    $language="es_ES.utf8";
putenv("LC_ALL=$language");
setlocale(LC_ALL, $language);
bindtextdomain("ssh", "./locale");
textdomain("ssh");

require("ssh_backend.php");

//Esta cosita sirve para poder cerrar una aplicacion, aunque
//las lazy sessions del shibboleth permitan el acceso
$GLOBALS['shib_Https'] = false;
$GLOBALS['shib_AssertionConsumerServiceURL'] =
"/federacion21.us.es/Shibboleth.sso";
$GLOBALS['shib_WAYF'] = "federacion21.us.es";

//si no esta autenticado, redirigimos al wayf
if( $_SERVER['HTTP_SHIB_IDENTITY_PROVIDER'] != "") {
        $_SESSION["user"] = $_SERVER['REMOTE_USER'];
}
else{
        $pageurl = "http://federacion21.us.es/protegido/ssh/ssh.php";
        $url = ($GLOBALS['shib_Https'] ? 'https' :  'http') .'://' .
                $GLOBALS['shib_AssertionConsumerServiceURL'] . "/WAYF/" .
                $GLOBALS['shib_WAYF'] .
                '?target=' . $pageurl;
        header("Location: ".$url);
}
?>

<html>
    <head>
        <title>SSH por federacion</title>
	<link rel="stylesheet" type="text/css" media="screen" href="style.css" />
    </head>
    <body>

	<div id="head"></div>
	<div id="main">
        <h2> <?php echo _('Bienvenido ') .
        htmlentities(get_remote_user()); ?></h2>
        <?php
            $var = anadir_usuario();
            $name = htmlentities(get_remote_user());

            if ($var == -1)
                echo '<p class="warning">'.
                _('No ha introducido un certificado valido.').'</p>';
            else if($var == -2)
                echo '<p class="warning">'.
                _('La operaci&oacute;n no se ha completado.').'</p>';

            else if($var == -3)
                echo '<p class="warning">'.
                _('No se ha facilitado un certificado, introduzcalo manualmente').'</p>';
            else
                echo '<p class="ok">'.
                _('Ahora puedes entrar por ssh en los servidores de la
                federaci&oacute;n, utilizando como nombre de usuario: 
                ').'<span class="user">'. $name .'</span></p>';
                
            $certificate = get_certificate_used($name);
            echo '<p>'._('Usando el certificado: ').'<br/>';
            $cert = str_split($certificate, 50);
	    echo '<div class="certificate">';
            foreach ($cert as $line){
                echo htmlentities($line)."<br/>";
            }
	    echo '</div>';
            echo '</p>';
        ?>

	<div class="info">
    <p>
<?php
            echo _('Si no est&aacute;s en tu puesto de trabajo,
            o no se encuentra tu clave, puedes proporcionar una manualmente,
            introducciendola en el siguiente campo. Mira en tu directorio
            $HOME/.ssh/id_rsa.pub');
?>
        </p>
        <p>
<?php
            echo _('Para utilizar el proceso autom&aacute;tico,
            ponte en contacto con tu proveedor de identidad.');
?>
        </p>
	</div>
	<div class="form">
            <?php display_form() ?>
	</div>
	</div>
    </body>
</html>

    \end{lstlisting}

    La aplicación de ejemplo de creación de cuentas, es una simple
    aplicación en php, que está basada en la anterior, por lo tanto
    comparte la mayor parte del estilo, y algo del código.

    Sin embargo, dado que es una aplicación de ejemplo, y su
    funcionamiento en teoría es muy simple, no se ha separado en
    varios ficheros, sino toda la aplicación está sobre un único
    fichero \texttt{useradd.php}.

    Esta aplicación hace uso de llamadas al sistema, para realizar
    comandos remotos en los servidores SSH registrados. Para que esto
    funcione, es necesario que el servidor donde se ejecute esta
    aplicación tenga las claves públicas de los demás servidores,
    puesto que se van a hacer llamadas a comandos remotos a través de
    SSH, y esto debería hacerse de manera automática.

    Veamos con más detalle cada una de las funciones que componen la
    pequeña aplicación de ejemplo.

    \begin{itemize}

    \item función \texttt{check\_user}

    \begin{lstlisting}

function check_user($uid, $servidor){
        if ($servidor == '')
                $servidor = "federacion21.us.es";

	exec("sudo ssh root@".$servidor." id ".$uid, $array, $retval);
	return $retval == 0;
}

    \end{lstlisting}

    Comprueba si un usuario tiene una cuenta ya creada en un servidor
    determinado. Para ello utiliza una llamada al sistema con el
    comando \texttt{ssh} hacia el servidor remoto, para ejecutar el comando
    \texttt{id}.

    \item función \texttt{useradd}

    \begin{lstlisting}

function useradd($uid, $servidor){
        if ($servidor == '')
                $servidor = "federacion21.us.es";

	$val = system("sudo ssh root@".$servidor." useradd -m -s /bin/bash -p xx -d /home/".$uid." ".$uid, $retval);
	return $retval;
}

    \end{lstlisting}

    Crea una cuenta para un usuario en un servidor remoto. Para ello,
    al igual que la función anterior utiliza el comando \texttt{ssh}
    para ejecutar el comando \texttt{useradd} en el servidor remoto.

    \end{itemize}
    
    El resto de la aplicación es trivial, y no tiene importancia
    comentarlo. Aquí se puede ver el código completo:

    \begin{lstlisting}

<?php 

$servers[0] = 'federacion21.us.es';
$desc[0] = "servidor1";
$servers[1] = 'federacion22.us.es';
$desc[1] = "servidor2";
$error = '';
$info = 'En esta p&aacute;gina podr&aacute;s solicitar la creaci&oacute;n
para el acceso ssh a los servidores';

session_start();
//soporte para traducciones
$lang=$_GET['lang'];
if($lang=='en')
$language="en_US.utf8";
else
$language="es_ES.utf8";
putenv("LC_ALL=$language");
setlocale(LC_ALL, $language);
bindtextdomain("ssh", "./locale");
textdomain("ssh");

//Esta cosita sirve para poder cerrar una aplicacion, aunque
//las lazy sessions del shibboleth permitan el acceso
$GLOBALS['shib_Https'] = false;
$GLOBALS['shib_AssertionConsumerServiceURL'] = "/federacion21.us.es/Shibboleth.sso";
$GLOBALS['shib_WAYF'] = "federacion21.us.es";

//si no esta autenticado, redirigimos al wayf
if( $_SERVER['HTTP_SHIB_IDENTITY_PROVIDER'] != "") {
        $_SESSION["user"] = $_SERVER['REMOTE_USER'];
}
else{
        $pageurl = "http://federacion21.us.es/protegido/ssh/useradd/";
        $url = ($GLOBALS['shib_Https'] ? 'https' :  'http') .'://' .
                $GLOBALS['shib_AssertionConsumerServiceURL'] . "/WAYF/" . $GLOBALS['shib_WAYF'] .
                '?target=' . $pageurl;
        header("Location: ".$url);
}

function useradd($uid, $servidor){
        if ($servidor == '')
                $servidor = "federacion21.us.es";

    $val = system("sudo ssh root@".$servidor." useradd -m -s /bin/bash -p xx -d /home/".$uid." ".$uid, $retval);
    return $retval;
}

function check_user($uid, $servidor){
        if ($servidor == '')
                $servidor = "federacion21.us.es";

    exec("sudo ssh root@".$servidor." id ".$uid, $array, $retval);
    return $retval == 0;
}

function get_remote_user(){
        $cad = $_SERVER["REMOTE_USER"];

        //$cad = $_SERVER["REMOTE_USER"];
        //partiendo el nombre, para crear danigm-us a partir de danigm@us.es
        $cad = $_SESSION['user'];
        $nado = explode('@', $cad);
        $name = $nado[0];
        $dominio = substr($nado[1], 0, stripos($nado[1], '.'));
        $name = $name.'-'.$dominio;
        return $name;
}
?>

<html>
    <head>
            <title>SSH por federacion</title>
            <link rel="stylesheet" type="text/css" media="screen" href="style.css" />
    <script type="text/javascript" src="js/jquery-1.2.1.min.js"></script>
    <script type="text/javascript" src="js/useradd.js"></script>

    </head>
    <body>

        <div id="head"></div>
        <div id="main">
            <h2> <?php echo _('Bienvenido ') . htmlentities(get_remote_user()); ?></h2>
            <?php
            if(isset($_GET['server'])){
                $s = $servers[$_GET['server']];
                $var = useradd(get_remote_user(), $s);
                if ($var != 0 && $var != 9)
                        $error = 'No ha sido posible crear la cuenta';
                else
                        $info = '<p>Se ha creado la cuenta correctamente en el servidor'.$s.'</p>';
            }
            //$var = useradd(get_remote_user(), 'federacion22.us.es');
            else{
                /*
                Mostrar lista de servidores
                disponibles, junto con enlace para crear cuenta.  Se crean por ssh, estando
                este servidor autorizado para entrar como root en los demas.  Quizas es
                mejor no hacerlo con usuario root, sino con un usuario que tenga permiso
                solo para el useradd.
                */
                echo "<h3>Servidores ssh ofrecidos por la Universidad de Sevilla</h3>";
                ?>
                <table>
                        <tr>
                                <th>Nombre</th>
                                <th>Descripci&oacute;n</th>
                                <th></th>
                        </tr>
                <?php
                for($i = 0; $i<count($servers); $i++){
                        echo '<tr>';
                        echo '<td>'.$servers[$i].'</td>';
                        echo '<td>'.$desc[$i].'</td>';
        if (check_user(get_remote_user(), $servers[$i]))
            echo '<td>Ya tienes cuenta en este servidor</td>';
        else
            echo '<td><a href="?server='.$uid.'" class="solicitar" id="'.$i.'" >solicitar cuenta</a></td>';
                        echo '</tr>';
                }
                echo "</table>";
        }
            ?>
            <div class="info">
                    <p class="warning">
                            <?php echo $error;?>
                    </p>
                    <?php echo $info ?>
            </div>
        </div>
    </body>
</html>

    \end{lstlisting}

    \subsection{Internacionalización}

    Puesto que estas aplicaciones son la parte que va a ver el
    usuario, se ha utilizado un sistema de internacionalización,
    \texttt{gettext}, para facilitar la traducción de las mismas. Así
    pues sería muy simple traducir estas aplicaciones y ofrecerlas en
    diferentes idiomas, según el idioma en el que esté configurado el
    navegador del usuario.

    De momento, y para este proyecto se ha realizado la versión en
    castellano, y una traducción al inglés.

    \section{Necesidades para montar la plataforma}

    Este proyecto se ha realizado con un software base determinado,
    puesto que depende de muchos componentes. En este apartado
    explicaremos cómo montar el proyecto desde cero, y tener una
    versión funcional.

    Como este proyecto depende de una federación de identidad,
    supondremos que pertenecemos a una organización de una federación,
    y por lo tanto tendremos diferentes SPs montados.

    Todo el código, el parche, y las dos aplicaciones en php, se
    pueden conseguir en la forja de rediris, en el proyecto AUPAAI 
    \url{https://forja.rediris.es/projects/aupaai/}.

    %TODO comentar cómo se consiguen los fuentes

    \begin{verbatim}

    svn checkout https://forja.rediris.es/svn/aupaai   
    cd aupaai/fed+ssh/

    \end{verbatim}

        \subsection{Cómo aplicar el parche}

        Para poder aplicar el parche, es necesario bajarse el código
        fuente de openssh, \url{http://www.openssh.org/portable.html}

        \begin{verbatim}

        export CVSROOT=anoncvs@anoncvs.mindrot.org:/cvs
        export CVS_RSH=/usr/bin/ssh
        cvs get openssh

        \end{verbatim}

        Una vez hecho esto, el siguiente paso es copiar el fichero
        .patch dentro del directio donde vayamos a compilar el
        servidor SSH, y aplicarlo.

        \begin{verbatim}
        
        cd openssh
        cp ../aupaai/fed+ssh/parche-openssh.patch .
        patch -p0 < parche-openssh.patch

        \end{verbatim}

        Por otra parte, para simplificar la instalación, y dado que la
        licencia BSD lo permite, está disponible el código parcheado,
        junto con las aplicaciones en la forja de rediris
        \url{http://forja.rediris.es/frs/download.php/775/openssh-federado.tar.gz}.

        Si todo ha ido bien, es hora de compilar, antes de nada, es
        necesario tener instalado el compilador, y todas las
        dependencias de openssh, además de las librerías openldap, ya
        que el parche hace uso de las mismas para comunicarse con el
        servidor de claves. Así pues se deben ejecutar los siguientes
        comandos.

        \begin{verbatim}

        ./configure --prefix=/ruta/de/instalacion
        make
        make install

        \end{verbatim}

        Una vez compilado, es hora de configurarlo para que funcione
        con nuestro sistema, cambiando el fichero \texttt{sshd\_config}

        \begin{verbatim}

        # External RSA key
        fedserver host.servidorclaves.com
        fedport 389
        usefed yes
        fedserver_root_dn "cn=admin,dc=us,dc=es"
        fedserver_root_pw password
        fedserver_base "o=People,dc=us,dc=es"
        fedserver_attr sshPublicKey
        fedserver_timeattr schacUserStatus

        \end{verbatim}

        Y con esto, ya estaría el servidor SSH parcheado, listo para
        funcionar de manera federada. Ahora sólo sería necesario crear
        las cuentas de los usuarios, que se puede hacer de forma
        dinámica, o estática, y sólo tendrán acceso cuando se
        autentiquen frente a la federación.

        \subsection{Cómo montar el SP y la aplicación web}

        La aplicación web de autenticación está hecha pensando en el
        SP de shibboleth 1.3. Por lo tanto será necesario tener un SP
        de shibboleth sobre un servidor web apache.

        Para funcionar, la aplicación requiere que estén instaladas
        las librerias de acceso a LDAP, para php, puesto que se va a
        comunicar con el servidor de claves, que en este caso será un
        servicio de directorio.
        
        Sobre un directorio protegido se despliega el directorio
        \texttt{ssh}, y ya estaría la aplicación de autenticación
        instalada en la federación.

        Cualquier usuario que quisiera acceder a algún servidor SSH
        federado, tendrá que entrar en esta página, de tal forma que
        escriba el registro correspondiente en el servidor de claves,
        y así el usuario tenga acceso a todos los servidores SSH
        federados.

        La aplicación es configurable, así que habrá que cambiar unas
        variables en el fichero \texttt{ssh\_backend.php}

        \begin{verbatim}

        $base_dn ='o=People,dc=us,dc=es';
        $servidor_ldap = "goonie.us.es";
        $puerto_ldap = 389;
        $bn = 'cn=admin,dc=us,dc=es';
        $pw = 'xxxx';
        $minutes_timeout = 30;
        $shib_header = "HTTP_USERCERTIFICATE";
        $rsa_server_key_attr = 'sshpublickey';
        $rsa_server_timeout = 'schacuserstatus';

        \end{verbatim}

        Con estas opciones se pueden configurar, el servidor de
        claves, el tiempo en minutos que va a durar una sesión de un
        usuario autenticado, la cabecera de shibboleth que mirará la
        aplicación para intentar conseguir la clave pública y los
        campos en el servidor de claves para almacenar la clave
        pública y el tiempo de sesión.

        \subsection{Cómo instalar el servidor de claves (openldap)}

        La última parte que hace falta para tener montado todo el
        proyecto, es el servidor de claves.

        Puesto que nos hemos decidido por un servicio de directorio,
        aquí es muy fácil la instalación, es más, hay diferentes
        opciones, comerciales, y no comerciales. Y dado que los
        campos a utilizar para almacenar los datos son variables,
        tanto en el servidor SSH parcheado, como en la aplicación de
        autenticación, no es obligatorio instalar ningún schema,
        aunque sí recomendable.

        Explicaremos aquí cómo instalar un servidor de claves, en una
        distribución tipo Debian, y aplicarle los esquemas usados.

        Lo primero es instalar el servicio de directorio.

        \begin{verbatim}

        apt-get install slapd

        \end{verbatim}

        La configuración está en el directorio /etc/ldap/, y los
        schemas para openldap hay que meterlos en /etc/ldap/schemas/.

        Descarga de los schemas necesarios:

        Schema schac: 
        \url{http://www.rediris.es/ldap/esquemas/schac/schac-20061017-1.3.0b2.schema.txt}

        Schema openssh-lpk:
        \url{http://dev.inversepath.com/openssh-lpk/openssh-lpk\_openldap.schema}


        Se ha de mover los schemas al directorio /etc/ldap/schemas/ y
        posteriormente modificar el fichero
        \texttt{/etc/ldap/slapd.conf} añadiendole las lineas

        \begin{verbatim}

        include         /etc/ldap/schema/openssh-lpk_openldap.schema
        include         /etc/ldap/schema/schac-20061017-1.3.0b2.schema

        \end{verbatim}

        
        \subsection{Todo en conjunto}

        Una vez que ya están todas las partes instaladas, hay que
        configurar tanto los servidores ssh parcheados, como la
        aplicación de autenticación para que se comuniquen con el
        servidor de claves públicas.

        %TODO poner algo más
