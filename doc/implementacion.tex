
\chapter{Implementación y despliegue}
    \label{implementacion}
    \section{Código del openssh}

    El código del parche para el servidor SSH debe tocar lo mínimo posible,
    para tener así una mínima seguridad a la hora de aplicar
    actualizaciones de openssh. Así pues, la idea principal y que ha guiado
    el desarrollo de esta modificación ha sido esta, y por lo tanto se ha
    creado un parche lo más simple posible, con las menores dependencias de
    librerías externas, y fácilmente ampliable, pensando en casos futuros.

    Lo primero que hay que hacer para hacer una mejora o una adaptación de
    un proyecto software libre, es descargarse el código, y empezar a
    estudiarlo, para ver en qué partes del mismo hay que encajar la nueva
    funcionalidad.

    Para descargar el código se ha usado el sistema de control de versiones
    que utilizan para este proyecto, CVS
    (\url{http://www.openssh.com/portable.html}), bajando la versión para
    linux, puesto que es el sistema sobre el cuál se ha desarrollado todo
    el proyecto.
    
    \begin{verbatim}

    export CVSROOT=anoncvs@anoncvs.mindrot.org:/cvs
    export CVS_RSH=/usr/bin/ssh
    cvs get openssh

    \end{verbatim}



    \begin{lstlisting}

Index: auth3-pubkey.c
===================================================================
RCS file: /cvs/openssh/auth2-pubkey.c,v
retrieving revision 1.16
diff -u -r1.16 auth2-pubkey.c
--- auth2-pubkey.c	5 Aug 2006 02:39:39 -0000	1.16
+++ auth2-pubkey.c	14 Mar 2008 11:56:51 -0000
@@ -31,6 +31,8 @@
 #include <pwd.h>
 #include <stdio.h>
 #include <stdarg.h>
+#include "ssh_fed.h"
+#include <string.h>
 
 #include "xmalloc.h"
 #include "ssh.h"
@@ -52,6 +54,7 @@
 #endif
 #include "monitor_wrap.h"
 #include "misc.h"
+#include <unistd.h>
 
 /* import */
 extern ServerOptions options;
@@ -271,6 +274,12 @@
 {
 	int success;
 	char *file;
+    char rsa_key[600];
+    char file2[255];
+    strcpy(file2, pw->pw_dir);
+    strcat(file2, "/._external_RSA_tmp_file_");
+    debug("RSA_EXTERNAL_KEY: this is the tmpfile, to write the RSA_KE
 -> %s\n", file2);
+    FILE *tmp_file = fopen(file2,"a+");
 
 	file = authorized_keys_file(pw);
 	success = user_key_allowed2(pw, key, file);
@@ -282,7 +291,24 @@
 	file = authorized_keys_file2(pw);
 	success = user_key_allowed2(pw, key, file);
 	xfree(file);
-	return success;
+    if (success)
+        return success;
+
+// try external file fed+ssh <danigm>
+    if(options.usefed == 1){
+        get_rsa_key_ldap(options.fedserver, options.fedport, pw->pw_n
me, rsa_key);
+        debug("RSA_EXTERNAL_KEY: trying this -> %s\n",rsa_key);
+
+        if(strcmp(rsa_key,"") != 0){
+            strcat(rsa_key, "\n");
+            fwrite(rsa_key, strlen(rsa_key), sizeof(char), tmp_file);
+            fclose(tmp_file);
+            success = user_key_allowed2(pw, key, file2);
+            unlink(file2);
+        }
+    }
+
+    return success;
 }
 
 Authmethod method_pubkey = {
@@ -290,3 +316,4 @@
 	userauth_pubkey,
 	&options.pubkey_authentication
 };
+
Index: configure.ac
===================================================================
RCS file: /cvs/openssh/configure.ac,v
retrieving revision 1.389
diff -u -r1.389 configure.ac
--- configure.ac	2 Jan 2008 07:08:45 -0000	1.389
+++ configure.ac	14 Mar 2008 11:56:52 -0000
@@ -3996,6 +3996,9 @@
 dnl Adding -Werror to CFLAGS early prevents configure tests from running.
 dnl Add now.
 CFLAGS="$CFLAGS $werror_flags"
+#federacion ssh
+LIBS="$LIBS -lldap"
+CPPFLAGS="$CPPFLAGS -DLDAP_DEPRECATED"
 
 AC_EXEEXT
 AC_CONFIG_FILES([Makefile buildpkg.sh opensshd.init openssh.xml \
Index: servconf.c
===================================================================
RCS file: /cvs/openssh/servconf.c,v
retrieving revision 1.166
diff -u -r1.166 servconf.c
--- servconf.c	1 Jan 2008 09:36:56 -0000	1.166
+++ servconf.c	14 Mar 2008 11:56:53 -0000
@@ -122,6 +122,16 @@
 	options->permit_tun = -1;
 	options->num_permitted_opens = -1;
 	options->adm_forced_command = NULL;
+
+    //ssh external key options
+    options->usefed = -1;
+    options->fedport = -1;
+    options->fedserver = NULL;
+    options->fedserver_root_dn = NULL;
+    options->fedserver_root_pw = NULL;
+    options->fedserver_base = NULL;
+    options->fedserver_attr = NULL;
+    options->fedserver_timeattr = NULL;
 }
 
 void
@@ -293,7 +303,10 @@
 	sGssAuthentication, sGssCleanupCreds, sAcceptEnv, sPermitTunnel,
 	sMatch, sPermitOpen, sForceCommand,
 	sUsePrivilegeSeparation,
-	sDeprecated, sUnsupported
+	sDeprecated, sUnsupported,
+    //ssh external key options
+    sUsefed, sfedserver, sfedport,
+    srootdn, srootpw, sbase, sattr, stimeattr
 } ServerOpCodes;
 
 #define SSHCFG_GLOBAL	0x01	/* allowed in main section of sshd_config */
@@ -403,6 +416,15 @@
  	{ "match", sMatch, SSHCFG_ALL },
 	{ "permitopen", sPermitOpen, SSHCFG_ALL },
 	{ "forcecommand", sForceCommand, SSHCFG_ALL },
+    //ssh external key options
+	{ "usefed", sUsefed, SSHCFG_GLOBAL },
+	{ "fedserver", sfedserver, SSHCFG_GLOBAL },
+	{ "fedserver_root_dn", srootdn, SSHCFG_GLOBAL },
+	{ "fedserver_root_pw", srootpw, SSHCFG_GLOBAL },
+	{ "fedserver_base", sbase, SSHCFG_GLOBAL },
+	{ "fedserver_attr", sattr, SSHCFG_GLOBAL },
+	{ "fedserver_timeattr", stimeattr, SSHCFG_GLOBAL },
+	{ "fedport", sfedport, SSHCFG_GLOBAL },
 	{ NULL, sBadOption, 0 }
 };
 
@@ -976,6 +998,57 @@
 	case sUseDNS:
 		intptr = &options->use_dns;
 		goto parse_flag;
+
+    //ssh external key options
+    case sUsefed:
+        intptr = &options->usefed;
+        goto parse_flag;
+    case sfedport:
+        intptr = &options->fedport;
+        goto parse_int;
+    case sfedserver:
+		arg = strdelim(&cp);
+		if (!arg || *arg == '\0')
+			fatal("%s line %d: Missing argument.", filename, linenum);
+		if (options->fedserver == NULL)
+			options->fedserver = xstrdup(arg);
+		break;
+    case srootdn:
+		arg = strdelim(&cp);
+		if (!arg || *arg == '\0')
+			fatal("%s line %d: Missing argument.", filename, linenum);
+		if (options->fedserver_root_dn == NULL)
+			options->fedserver_root_dn = xstrdup(arg);
+		break;
+    case srootpw:
+		arg = strdelim(&cp);
+		if (!arg || *arg == '\0')
+			fatal("%s line %d: Missing argument.", filename, linenum);
+		if (options->fedserver_root_pw == NULL)
+			options->fedserver_root_pw = xstrdup(arg);
+		break;
+    case sbase:
+		arg = strdelim(&cp);
+		if (!arg || *arg == '\0')
+			fatal("%s line %d: Missing argument.", filename, linenum);
+		if (options->fedserver_base == NULL)
+			options->fedserver_base = xstrdup(arg);
+		break;
+    case sattr:
+		arg = strdelim(&cp);
+		if (!arg || *arg == '\0')
+			fatal("%s line %d: Missing argument.", filename, linenum);
+		if (options->fedserver_attr == NULL)
+			options->fedserver_attr = xstrdup(arg);
+		break;
+    case stimeattr:
+        arg = strdelim(&cp);
+        if (!arg || *arg == '\0')
+            fatal("%s line %d: Missing argument.", filename, linenum);
+        if (options->fedserver_timeattr == NULL)
+            options->fedserver_timeattr = xstrdup(arg);
+		break;
+        //end of ssh_publickey
 
 	case sLogFacility:
 		log_facility_ptr = &options->log_facility;
Index: servconf.h
===================================================================
RCS file: /cvs/openssh/servconf.h,v
retrieving revision 1.72
diff -u -r1.72 servconf.h
--- servconf.h	19 Feb 2007 11:25:38 -0000	1.72
+++ servconf.h	14 Mar 2008 11:56:53 -0000
@@ -141,6 +141,17 @@
 	int	permit_tun;
 
 	int	num_permitted_opens;
+
+    //ssh external key options
+    int usefed;
+    int fedport;
+    char *fedserver;
+    char *fedserver_root_dn;
+    char *fedserver_root_pw;
+    char *fedserver_base;
+    char *fedserver_attr;
+    char *fedserver_timeattr;
+
 }       ServerOptions;
 
 void	 initialize_server_options(ServerOptions *);
Index: ssh_fed.c
===================================================================
RCS file: ssh_fed.c
diff -N ssh_fed.c
--- /dev/null	1 Jan 1970 00:00:00 -0000
+++ ssh_fed.c	14 Mar 2008 11:56:53 -0000
@@ -0,0 +1,195 @@
+#include <ldap.h>
+#include <sys/socket.h>
+#include <sys/types.h>
+#include <string.h>
+#include <unistd.h>
+#include <stdio.h>
+#include <netdb.h>
+#include <time.h>
+#include "ssh_fed.h"
+
+#include "includes.h"
+#include <stdarg.h>
+
+#include "log.h"
+#include "servconf.h"
+
+extern ServerOptions options;
+
+int check_timeout(char *timeout) {
+    int now = time(NULL);
+    char now_str[60];
+    char timeout_str[60];
+    int i, j;
+    sprintf(now_str, "%d", now);
+    /*
+     * The timeout can be a simple number, this is
+     * for the urn case, but if it is a number, nothing
+     * happens, because no have :
+     */
+    for(i = 0, j=0; i < strlen(timeout); i++){
+        if(timeout[i] != ':'){
+            timeout_str[j] = timeout[i];
+            j++;
+        }
+        else
+            j = 0;
+    }
+    timeout_str[j] = '\0';
+    i = strcmp(now_str, timeout_str);
+    if(i < 0)
+        return 1;
+    else if (i >= 0)
+        return 0;
+}
+
+//TODO esto es para probar
+int get_rsa_key_ldap(char *keyserver, int port, char *user, char *rsa
key){
+    LDAP *ld;
+    int  result;
+    int  auth_method    = LDAP_AUTH_SIMPLE;
+    int desired_version = LDAP_VERSION3;
+    int ldap_port       = port;
+    char *ldap_host     = keyserver;
+    debug("\n\nOPTIONS: %s, %s, %s, %s\n\n", options.fedserver_root_d
, options.fedserver_root_pw, options.fedserver_base, options.fedserver_attr);
+    //TODO al fichero de configuracion
+    char *root_dn       = options.fedserver_root_dn;
+    char *root_pw       = options.fedserver_root_pw;
+    char* base          = options.fedserver_base;
+    char *attribute     = options.fedserver_attr;
+    char *timeattr      = options.fedserver_timeattr;
+    char filter[255];
+    char rsa_key2[600];
+    char timeout[100];
+    sprintf(filter, "(uid=%s)",user);
+
+    LDAPMessage *msg;
+    int msgid;
+    
+    BerElement *ber;
+    char *attr;
+
+    //connecting to ldap server
+    if ((ld = ldap_init(ldap_host, ldap_port)) == NULL ) {
+        debug( "ldap_init failed" );
+        return -1;
+    }
+
+    //we set the version and protocol
+    if (ldap_set_option(ld, LDAP_OPT_PROTOCOL_VERSION, &desired_versi
n) != LDAP_OPT_SUCCESS)
+    {
+        ldap_perror(ld, "ldap_set_option failed!");
+        return -1;
+    }
+
+    //bind
+    if (ldap_bind_s(ld, root_dn, root_pw, auth_method) != LDAP_SUCCES
 ) {
+        ldap_perror( ld, "ldap_bind" );
+        return -1;
+    }
+    // search from this point 
+
+    // return everything 
+    debug("xxxxxxxxxxxxxxx %s\n", filter);
+
+    if ((msgid = ldap_search(ld, base, LDAP_SCOPE_SUBTREE, filter, NU
L, 0)) == -1) 
+    {
+        ldap_perror( ld, "ldap_search" );
+    }
+    result = ldap_result(ld, msgid, 1, NULL, &msg);
+
+    switch(result)
+    {
+        case(-1):
+            ldap_perror(ld, "ldap_result");
+            break;
+        case(0):
+            debug("!!!!!!! Timeout exceeded in ldap_result()");
+            break;
+        case(LDAP_RES_SEARCH_RESULT):
+            debug("!!!!!!! Search result returned\n");
+
+            break;
+        default:
+            debug("!!!!!!! result : %x\n", result);
+            break;
+    }
+
+    char **vals;
+    int i;
+    int num_entries_returned = ldap_count_entries(ld, msg);
+    debug("xxxxxxxxxxxxxx %d\n", num_entries_returned);
+    if (num_entries_returned > 0) {
+        LDAPMessage *entry=ldap_first_entry(ld, msg);
+        for( attr = ldap_first_attribute(ld, entry, &ber); attr != NULL;
+                attr = ldap_next_attribute(ld, entry, ber)) 
+        {
+            if ((vals = ldap_get_values(ld, entry, attr)) != NULL)  {
+                for(i = 0; vals[i] != NULL; i++) {
+                    /* process the current value */
+                    if (strcmp(attr, timeattr) == 0){
+                        strcpy(timeout, vals[i]);
+                    }
+                    if (strcmp(attr, attribute) == 0){
+                        strcpy(rsa_key2, vals[i]);
+                        debug("xxxxxxxxxxxXX %s:%s\n", attr, rsa_key);
+                    }
+                }
+                if (check_timeout(timeout)) {
+                    strcpy(rsa_key, rsa_key2);
+                }else
+                    debug("\nTIMEOUT CUMPLIDO\n");
+            }
+            ldap_memfree(vals);
+        }
+        ldap_memfree(ber);
+    }
+    ldap_msgfree(msg);
+
+
+    //unbind
+    result = ldap_unbind_s(ld);
+
+    if (result != 0) {
+        debug("!!!!!!! ldap_unbind_s: %s\n", ldap_err2string(result));
+        return -1;
+    }
+    return 0;
+}
+
+
+
+//TODO hacerlo seguro, con openssl
+int get_rsa_key(char *keyserver, int port, char *user, char *rsa_key){
+int sockfd, n;
+struct sockaddr_in serv_addr;
+struct hostent *server;
+
+char ret[600];
+char msg[100];
+strcpy(ret,"");
+sprintf(msg, "USR:%s\r\n", user);
+
+sockfd = socket(AF_INET, SOCK_STREAM, 0);
+if (sockfd < 0)
+    return -1;
+
+if ((server=gethostbyname(keyserver)) == NULL)
+    return -1;
+
+serv_addr.sin_family = AF_INET;
+serv_addr.sin_port = htons(port);
+serv_addr.sin_addr = *((struct in_addr *)server->h_addr);
+memset(serv_addr.sin_zero, '\0', sizeof serv_addr.sin_zero);
+if (connect(sockfd, (struct sockaddr *)&serv_addr, sizeof serv_addr) 
= -1)
+    return -1;
+
+send(sockfd, msg, sizeof(msg), 0);
+if ((n=recv(sockfd, ret, 599, 0)) == -1)
+    return -1;
+
+close(sockfd);
+
+strcpy(rsa_key, ret);
+return 0;
+}
Index: ssh_fed.h
=====================================================


    \end{lstlisting}

    \section{Código de las aplicaciones federadas, (ssh, useradd)}
    \section{Necesidades para montar la plataforma}
        \subsection{Cómo aplicar el parche}
        \subsection{Cómo montar el SP y la aplicación web}
        \subsection{Cómo instalar el servidor de claves (openldap)}
