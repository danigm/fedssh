\section{Introducción}
\subsection{Objetivos}

Las facilidades que ofrece la federación de identidad son más que
evidentes, y podría ser interesante en muchos casos, tener estas
facilidades para otros servicios que requieran autenticación.

El principal objetivo de este proyecto es llevar las facilidades de la
gestión de identidad del ámbito de la web a otros servicios, como por
ejemplo el SSH. En este proyecto nos hemos centrado en integrar la
federación de identidad con el acceso por SSH, y puede servir como
prueba de concepto a la hora de llevar la autenticación por federación
de identidad a servicios diferentes de la web.

Las características más importantes de la federación de identidad, que
nos serán útiles en el SSH son:
\begin{enumerate}

    \item \textbf{Acceso a recursos de otras entidades}: La base de la
    federación de identidad es poder acceder a recursos de otra
    entidad con la misma cuenta con la que accedes a los recursos o
    servicios de tu propia entidad.

    \item \textbf{Gestión de identidad distribuida}: Al encargarse
    cada entidad de la federación de sus propios usuarios, y
    basándonos en las relaciones de confianza de la federación, se
    puede dar servicio a un mayor número de usuarios gestionando tan
    solo una pequeña cantidad de ellos. Esto puede crear algún tipo de
    duda, puesto que se pierde el control sobre los usuarios, pero no
    hay que olvidar que la federación es una red de confianza, donde
    cada entidad debe confiar en las demás, y para ello hay mecanismos
    seguros, como por ejemplo los certificados.

    \item \textbf{Unicidad de contraseña}: La federación de identidad
    nos brinda la posibilidad de acceder a diferentes servicios, que
    requieren autenticación, con la misma cuenta y la misma
    contraseña, y sin necesidad de replicar esta en los diferentes
    servicios, sino estando en tu propia entidad, incrementando así la
    seguridad de la misma, y la comodidad a la hora de cambiar de
    contraseña, o de nombre de usuario.

    \item \textbf{Login único}: También se busca implementar el Single
    Sing On(SSO) para el SSH sobre federación, de tal forma que un
    usuario sólo tenga que autenticarse una vez, y a partir de ahí,
    tener acceso, sin necesidad de introducir ningún tipo de
    contraseña, a todos los servidores SSH disponibles.

\end{enumerate}

Por otra parte, hemos elegido llevar la federación al servicio SSH
porque es ampliamente utilizado, además de que ofrece una gran
potencia y versatilidad, abriendo así la puerta a la utilización de
otros servicios de forma fácil.

Por ejemplo, en el ambiente académico, puede ser interesante dar
acceso a un servidor SSH a todos los alumnos de Informática, bien sea
para que utilicen un supercomputador, o para que tengan una cuenta
dónde hacer las practicas. Dentro del objetivo de este proyecto
entraría delegar la gestión de estos usuarios a la federación,
facilitar así el proceso, así como por parte del alumno, como por
parte del administrador de las máquinas.
%TODO importancia del SSH, túneles, importación X

\subsection{Antecedentes}
    (feide, parche opensshldap)


