\chapter{Conclusiones}

Este proyecto ha supuesto para mi una experiencia nueva, puesto que no se
trata de un desarrollo propio, sino que extender algo ya existente, para
adaptarlo a unas necesidades nuevas. Además he presenciado cómo se ha
creado, y cómo se está creando una federación de identidad desde dentro, a
todos los niveles, desde el nivel técnico, al nivel institucional.

Al principio puede parecer un proyecto muy complejo, puesto que no se
conoce el código del servidor SSH, y realizar una adaptación de algo tan
grande puede resultar abrumador. Gracias a la claridad del código, y a la
separación del mismo, he conseguido realizar la modificación sin demasiadas
complicaciones.

Puedo concluir de esta experiencia, que las licencias libres, ayudan a la
innovación y aprendizaje, ya que te puedes basar en el código y
conocimiento de otras personas, para mejorar y realizar tus proyectos.

Hoy en día, cualquier organización maneja una gran cantidad de máquinas, y
la mayoría se manejarán de manera remota. Para el acceso remoto lo más
rápido y sencillo es utilizar el protocolo SSH, ya que proporciona
seguridad, y flexibilidad.

Por otra parte, cada vez más se está imponiendo la federación de identidad,
tanto en el ámbito universitario, como en el empresarial, por lo tanto la
federación de identidad es una tecnología a tener en cuenta para el futuro
próximo.

Este proyecto hace que la gestión de máquinas de manera remota se
simplifique, al utilizar los conceptos de la federación de identidad en el
acceso por SSH, por lo tanto simplifica la gestión de multitud de máquinas,
de igual manera que la federación de identidad simplifica la gestión de los
usuarios a las aplicaciones web.

Con esta idea de gestión de identidad para el acceso SSH, el permitir o
prohibir el acceso a un servicio SSH es tan simple como visitar una página
web.

Además este proyecto abre la puerta para la federación de identidad en
aplicaciones que no sean web. Este proyecto es una prueba de que es posible
utilizar los mecanismos de federación de identidad para autenticar a
usuarios en una aplicación que utiliza un navegador como medio.

%TODO el resultado de este proyecto puede resultar una aportación
%relevante al uso de la federación.
Por todas estas razones pienso que este proyecto puede ser un comienzo, o
algo importante en esto de la federación de identidad, que a día de hoy se
está imponiendo, puesto que cada vez hay más usuarios, y esto simplifica la
gestión de la identidad de los mismos.
