\section{Resumen}
Cada vez está tomando más importancia la web, y actualmente están
apareciendo multitud de aplicaciones, ya sea por parte de empresas o
de instituciones, como Universidades, etc. Estas aplicaciones
normalmente requieren autenticación, y la mayoría de ellas basan dicha
autenticación en bases de datos locales.

Con el crecimiento de las aplicaciones, crece el número de usuarios de
estas, y por parte del usuario, crece el número de cuentas creadas
para diferentes aplicaciones. Para paliar este problema, nace el
Single Sing On(SSO), que proporciona una única cuenta, y un único
punto de autenticación, para diferentes aplicaciones web de una misma
entidad, por ejemplo la Universidad de Sevilla.

El siguiente paso natural, es el uso de aplicaciones de otras
entidades, y aquí es donde radica la importancia de la
\textbf{federación de identidad}.

La federación de identidad consiste en que una serie de entidades
\textbf{confían} en otras para la autenticación de los usuarios. Es
decir, que una aplicación de una entidad, acepta usuarios de otra
entidad. Además estos usuarios se autenticarán en su entidad, por lo
que la gestión de usuarios, contraseñas, y atributos, queda delegada a
cada entidad. Por tanto el usuario final tiene acceso a todas las
aplicaciones federadas, de todas las entidades que conforman la
federación de identidad.

Esto facilita enormemente la gestión de usuarios, por parte de las
entidades, puesto que tan solo tienen que gestionar sus propios
usuarios, y prestan servicio a usuarios de otras entidades, gracias a
que la autenticación es delegada.

La idea principal de este proyecto es llevar las facilidades que
proporciona la federación de identidad fuera del ámbito de la web,
más concretamente al ámbito del \textbf{SSH} (Secure SHell).

Para el caso del SSH, si un usuario tiene acceso a diferentes
máquinas, tendrá diferentes cuentas, y diferentes contraseñas que
recordar, almacenar y gestionar, con la problemática que eso conlleva.
Además sus contraseñas estarán en máquinas que no tiene por qué
controlar, por lo que si se compromete alguna de estas máquinas,
estará comprometida su contraseña.

Utilizando SSH sobre la federación de identidad, se pueden eliminar
estos problemas e incrementar la comodidad, tanto por parte de los
usuarios, como por parte de los administradores. Para poder acceder
por SSH, un usuario tendría que autenticarse en la federación, y una
vez autenticado, podrá entrar en todas las máquinas que ofrezcan el
servicio de SSH federado, sin necesidad de poner contraseña, basándose
en el mecanismo de clave pública y clave privada, y estando el
servidor en cualquier entidad de la federación.

Así pues, por parte del usuario, se tiene acceso a diferentes máquinas
necesitando recordar y gestionar una sola contraseña. Además esta
contraseña nunca se entrega a un servidor extraño, sólo a la entidad
de la cual procedes (a través de una páginas web segura), y en la cual
confías puesto que es la encargada de gestionar tu identidad.

Y por parte del administrador, se puede delegar la gestión de
usuarios, confiando en la federación. Automatizando la creación y
destrucción de cuentas, y sin necesidad de proporcionar ninguna
contraseña.
