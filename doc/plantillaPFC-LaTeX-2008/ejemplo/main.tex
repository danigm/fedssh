\documentclass{memoriaPFC}

%
% DATOS DEL DOCUMENTO
%

\title{Robopsicología aplicada a nuevos modelos de redes positrónicas}

%\autores{Nombre1 Apellidos1}{Nombre2 Apellidos2}
\autores{Susan Calvin}{}

\director{Isaac Asimov}
%\directora{Ursula K. Le Guin}

% iaei ii ioi itel
\titulacion{ii}

% mayo || septiembre (en minúsculas!!)
\date{mayo de 2007}

\resumen{%
Esto es el resumen. Esto es el emocionante resumen. Esto es el resumen. Esto es el emocionante resumen. Esto es el resumen. Esto es el emocionante resumen. Esto es el resumen. Esto es el emocionante resumen. Esto es el resumen. Esto es el emocionante resumen. Esto es el resumen. Esto es el emocionante resumen. Esto es el resumen. Esto es el emocionante resumen. Esto es el resumen. Esto es el emocionante resumen. Esto es el resumen. Esto es el emocionante resumen. Esto es el resumen. Esto es el emocionante resumen. Esto es el resumen. Esto es el emocionante resumen. 
}
\descriptores{Robopsicología, robótica, psicología, informática, ciencia-ficción}


%
% COMIENZO DEL DOCUMENTO
%
\begin{document}

% Portada, resumen, indices
\frontmatter
\hacerportada
\hacerresumen
\tableofcontents
\listoffigures % Opcional
\listoftables % Opcional
\lstlistoflistings % Opcional

% Contenidos
\mainmatter
\chapter{Introducción}
\input{dummy}

\chapter{Objetivos}

\chapter{Conclusiones}

% Apéndices
\backmatter
\appendix
\chapter{Acrónimos}

\chapter{Licencia}
\input{0202_apendice}

\bibliografia{referencias}
\bibliografiaOtras{otrasreferencias} %Opcional

\chapter{Agradecimientos}

\begin{itemize}
 \item A Borja Sotomayor, que sin su plantilla de 2003 hubiera sido bastante más difícil conseguir una para este año.
 \item A Rebeca Cortázar, por la ayuda prestada para que esta plantilla cumpliese las normas de entrega.
 \item A Pablo y Unai.
 \item A Javier y Juan José, del grupo de usuarios hispanohablantes de \TeX{}.
\end{itemize}

\end{document}
